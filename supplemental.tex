\documentclass[12pt]{article}

\usepackage{amsmath}
\usepackage{graphicx}
\usepackage{enumerate}
\usepackage{natbib}

% DON'T change margins - should be 1 inch all around.
\addtolength{\oddsidemargin}{-.5in}%
\addtolength{\evensidemargin}{-1in}%
\addtolength{\textwidth}{1in}%
\addtolength{\textheight}{1.7in}%
\addtolength{\topmargin}{-1in}%

%% ABOVE THIS LINE IS JASA TEMPLATE
%% BELOW THIS LINE IS MY STUFF

% add bibliography to TOC - remove for final version
% \usepackage[nottoc,numbib]{tocbibind}

\usepackage[utf8]{inputenc}
\usepackage{amssymb}
\usepackage{floatpag}
\usepackage{bm}
% \usepackage{algorithm2e}
\usepackage[unicode,psdextra]{hyperref}
\usepackage[nameinlink,capitalise,noabbrev]{cleveref}
\usepackage{csquotes}
\usepackage[T1]{fontenc}
\usepackage{textcomp} % provide symbols
\usepackage{xr}
\usepackage{afterpage}
\usepackage{caption}
\usepackage{numprint}
\npfourdigitnosep
\npdecimalsign{.}
% \usepackage{microtype}

% Generic maths commands
\def\reals{\mathbb{R}}
\def\nats{\mathbb{N}}
\def\sampSpace{\mathcal{X}}
\def\dist{\sim}
\DeclareMathOperator{\E}{\mathbb{E}}
\DeclareMathOperator{\V}{\mathbb{V}}
\DeclareMathOperator{\I}{\mathbb{I}}
%\DeclareMathOperator{\prob}{\mathbb{P}}
\newcommand{\ind}{\mathrel{\perp\!\!\!\perp}}
\DeclareMathOperator{\prob}{\mathrm{Pr}}
\DeclareMathOperator{\p}{\pi}
\DeclareMathOperator{\var}{\mathbb{V}}
\DeclareMathOperator{\indicator}{\mathbb{I}}
\DeclareMathOperator{\cov}{Cov}
\DeclareMathOperator{\cor}{Cor}
\DeclareMathOperator{\logit}{logit}
\DeclareMathOperator{\Ber}{Bernoulli}
\DeclareMathOperator{\Bin}{Binomial}
\DeclareMathOperator{\Poi}{Poisson}
\DeclareMathOperator{\BetaDist}{Beta}
\DeclareMathOperator{\Exponential}{Exponential}
\DeclareMathOperator{\NBr}{NegBin}
\newcommand{\NBc}{\NBr}
% \newcommand{\NBs}{\NBr_{s}}
\DeclareMathOperator{\BB}{BetaBin}
\DeclareMathOperator{\GamDist}{Gamma}
\DeclareMathOperator{\MN}{Multinomial}
\DeclareMathOperator{\N}{N}
\DeclareMathOperator{\MNorm}{N}
\DeclareMathOperator{\LN}{LN}
\DeclareMathOperator{\LKJ}{LKJ}
\DeclareMathOperator{\expit}{expit}
\newcommand\matr{\bm}
\newcommand\set{\mathcal}
\renewcommand{\vec}[1]{\bm{#1}}
\newcommand{\ssep}{:}
\DeclareMathOperator*{\argmax}{arg\,max}

\newcommand\citePersonalComms[1]{(#1, personal communication)}

% Thesis-specific maths commands
\newcommand{\dmax}{d_\text{max}}
\newcommand{\psens}{p_\text{sens}}
\newcommand{\psenss}{p_\text{sens}^{(s)}}
\newcommand{\psensi}{p_\text{sens}^{(i)}}
\newcommand{\ntot}{n_\text{tot}}
\newcommand{\ndet}{n_\text{d}}
\newcommand{\nnodet}{n_\text{u}}
\newcommand{\pnodet}{p_\text{u}}
\newcommand{\Npop}{N_\text{pop}}
\newcommand{\Ncis}{N_\text{CIS}}
\newcommand{\ncis}{\vec{n_\text{CIS}}}
\newcommand{\na}{\vec{n}_\text{obs}}
\newcommand{\pcis}{\vec{p_\text{CIS}}}
%\newcommand{\sched}{\mathbb{T}}
%\newcommand{\nsched}{n_{\sched}}
\newcommand{\sched}{\mathcal{T}}
\newcommand{\nsched}{|\sched{}|}
\newcommand{\inform}{{_{\text{inform}}}}
\newcommand{\posResults}{r_{+}}
\newcommand{\negResults}{r_{-}}


%% Bibliography
% \usepackage[authordate-trad,backend=biber]{biblatex-chicago}
% \addbibresource{references.bib}

% Macros for common abbreviations to get the spacing right
% See: https://stackoverflow.com/questions/3282319/correct-way-to-define-macros-etc-ie-in-latex
\usepackage{xspace}
\makeatletter
\DeclareRobustCommand\onedot{\futurelet\@let@token\@onedot}
\def\@onedot{\ifx\@let@token.\else.\null\fi\xspace}
\def\eg{e.g\onedot} \def\Eg{{E.g}\onedot}
\def\ie{i.e\onedot} \def\Ie{{I.e}\onedot}
\def\cf{c.f\onedot} \def\Cf{{C.f}\onedot}
\def\etc{etc\onedot} \def\vs{{vs}\onedot}
\def\wrt{w.r.t\onedot} \def\dof{d.o.f\onedot}
\def\etal{et al\onedot}
\makeatother

%% LINE NUMBERS
\usepackage{lineno} % Include the package for line numbering
\linenumbers % Activates line numbering for the document


%----Helper code for dealing with external references----
% (by cyberSingularity at http://tex.stackexchange.com/a/69832/226)

\usepackage{xr}
\makeatletter

\newcommand*{\addFileDependency}[1]{% argument=file name and extension
\typeout{(#1)}% latexmk will find this if $recorder=0
% however, in that case, it will ignore #1 if it is a .aux or 
% .pdf file etc and it exists! If it doesn't exist, it will appear 
% in the list of dependents regardless)
%
% Write the following if you want it to appear in \listfiles 
% --- although not really necessary and latexmk doesn't use this
%
\@addtofilelist{#1}
%
% latexmk will find this message if #1 doesn't exist (yet)
\IfFileExists{#1}{}{\typeout{No file #1.}}
}\makeatother

\newcommand*{\myexternaldocument}[1]{%
\externaldocument{#1}%
\addFileDependency{#1.aux}%
\addFileDependency{#1.tex}%
}
\myexternaldocument{main}
%------------End of helper code--------------


\begin{document}

%\bibliographystyle{natbib}

\def\spacingset#1{\renewcommand{\baselinestretch}%
{#1}\small\normalsize} \spacingset{1}


%%%%%%%%%%%%%%%%%%%%%%%%%%%%%%%%%%%%%%%%%%%%%%%%%%%%%%%%%%%%%%%%%%%%%%%%%%%%%%


\appendix

\section{Derivations of quantities in \cref{sec:inference}} \label{sec:derivations}

\subsection{Expressions in \cref{sec:inference}}

First, the derivation of $p(\vec{\theta} \mid \na)$:
\begin{align}
p(\vec{\theta} \mid \na)
&\propto p(\vec{\theta}) p(\na \mid \vec{\theta}) \\
&= p(\vec\theta) \sum_{\ntot= \ndet}^{\infty} p(\ntot, \na \mid \vec{\theta}) \\
&= p(\vec{\theta}) \sum_{\ntot=\ndet}^\infty p(\ntot \mid \vec{\theta}) p(\na \mid \ntot, \vec{\theta}) \\
&= p(\vec{\theta}) \sum_{\ntot=\ndet}^\infty p(\ntot \mid \vec{\theta}) \frac{\ntot!}{(\ntot - \ndet)!} \pnodet^{\ntot - \ndet} \prod_{k \in \set{D}} p_k &\text{by \cref{perf-test:eq:multinomial}} \\
&= p(\vec{\theta}) \left( \prod_{k \in \set{D}} p_k \right) \left( \sum_{\ntot=\ndet}^\infty p(\ntot \mid \vec{\theta}) \frac{\ntot!}{(\ntot - \ndet)!} \pnodet^{\ntot - \ndet} \right).
\intertext{For convenience, define the summation term as:}
\eta &= 
\sum_{\ntot=\ndet}^\infty p(\ntot \mid \vec{\theta}) \frac{\ntot!}{(\ntot - \ndet)!} \pnodet^{\ntot - \ndet}. \label{perf-test:eq:eta}
\end{align}

Next, we derive an analytical solution to $\eta$ (defined in \cref{perf-test:eq:eta}) assuming the prior $\ntot \dist \NBc(\mu, r)$, and that it is independent of $\vec{\theta}$, the parameters of the survival distribution.
Therefore, $p(\ntot \mid \vec{\theta}) = p(\ntot)$.
This assumption makes $\eta$ analytically tractable, allowing computationally feasible inference.

Putting a negative binomial prior on $\ntot$ is equivalent to the following gamma-Poisson composite; its use simplifies the derivation.
\begin{align}
\ntot \mid \lambda &\dist \Poi(\lambda) \\
\lambda &\dist \GamDist(a, b)
\end{align}
where $b = r / \mu$ and $a = r$.
Hence:
\begin{align}
\eta
&= \int \sum_{\ntot=\ndet}^\infty \frac{\ntot!}{(\ntot-\ndet)!} \pnodet^{\ntot-\ndet} p(\ntot \mid \lambda) p(\lambda) d\lambda &\text{$\lambda$ explicit}\\
&= \int \sum_{\ntot=\ndet}^\infty \frac{\ntot!}{(\ntot-\ndet)!} \pnodet^{\ntot-\ndet} \frac{\lambda^{\ntot} e^{-\lambda}}{\ntot!} p(\lambda) d\lambda &\ntot \dist \Poi\\
%&= \int \sum_{\ntot=\ndet}^\infty \frac{1}{(\ntot-\ndet)!} \pnodet^{\ntot-\ndet} \lambda^{\ntot-\ndet} \lambda^{\ndet} e^{-\lambda} p(\lambda) d\lambda \\
&= \int \lambda^{\ndet} e^{-\lambda} p(\lambda) \sum_{\nnodet=0}^\infty \frac{(\pnodet \lambda)^{\nnodet}}{\nnodet!} d\lambda &\nnodet = \ntot-\ndet\\
&= \int \lambda^{\ndet} e^{-\lambda} p(\lambda) e^{\lambda \pnodet} d\lambda &\text{Maclaurin series of $e$} \\
&= \int \lambda^{\ndet} e^{-\lambda(1 - \pnodet)} \frac{b^a}{\Gamma(a)} \lambda^{a-1} e^{-b\lambda} d\lambda &\lambda \dist \GamDist\\
&= \int \frac{b^a}{\Gamma(a)} \lambda^{a+\ndet-1} e^{-(b+1-\pnodet)\lambda} d\lambda \\
&= \frac{b^a}{\Gamma(a)} \frac{\Gamma(a+\ndet)}{(b+1-\pnodet)^{a+\ndet}} &\text{Gamma pdf}\\
&\propto (b+1-\pnodet)^{-(a+\ndet)} &\text{only $p_u$ depends on $\theta$}\\
&= (r/\mu + 1 - \pnodet)^{-(r+\ndet)} &\text{sub in $\mu$ and $r$}\\
&\propto(r + \mu (1- \pnodet))^{-(r+\ndet)}.
\end{align}

\subsection{Expressions in \cref{sec:prob-undetected}}

If $i_j = i_k$ then the event $O_j = \vec{\nu}_k$ occurs if and only if the episode starts in the interval $[l^{(b)}_k, r^{(b)}_k]$ and ends in the interval $[l^{(e)}_k, r^{(e)}_k]$.
 on $\vec{\theta}$ and $i_j = i_k$, this gives:
\begin{align}
p_{ik}
=& \prob \left( l_k^{(b)} \leq B_{j} \leq r_k^{(b)}, l_k^{(e)} \leq E_{j} \leq r_k^{(e)} \right) \\
=& \prob \left( l_k^{(e)} \leq E_{j} \leq r_k^{(e)} \mid l_k^{(b)} \leq B_{j} \leq r_k^{(b)} \right) \times\prob \left( l_k^{(b)} \leq B_{j} \leq r_k^{(b)} \right) \\
=& \sum_{b = l_k^{(b)}}^{r_k^{(b)}} \prob \left( l_k^{(e)} \leq E_{j} \leq r_k^{(e)} \mid B_{j} = b \right) \prob \left(B_{j} = b \right) \\
=& \sum_{b = l_k^{(b)}}^{r_k^{(b)}} \prob \left( l_k^{(e)} - b + 1 \leq D_{j} \leq r_k^{(e)} - b + 1 \right) \prob \left(B_{j} = b \right) &\text{by def of $D_{j}$} \\
=& \sum_{b = l_k^{(b)}}^{r_k^{(b)}} \left( S_{\vec{\theta}}(l_k^{(e)} - b + 1) - S_{\vec{\theta}}(r_k^{(e)} - b + 2) \right) \prob \left(B_{j} = b \right) &\text{by def of $S_{\vec{\theta}}$} \\
\propto& \sum_{b = l_k^{(b)}}^{r_k^{(b)}} \left( S_{\vec{\theta}}(l_k^{(e)} - b + 1) - S_{\vec{\theta}}(r_k^{(e)} - b + 2) \right)
\end{align}
under the assumption of uniform probability of infection time.
This is the standard form of the likelihood for doubly interval censored data without truncation~\citep[e.g.][]{sunEmpirical}.

\subsection{Expression for $p_{u}$}

\begin{align}
  1 - p_u
  &= 1 - \sum_{i=1}^{\Ncis} \prob(O_j = \varnothing, i_j = i \mid \vec{\theta}) \\
  &= 1 - \sum_{i=1}^{\Ncis} \prob(O_j = \varnothing \mid i_j = i, \vec{\theta}) P(i_j = i \mid \vec{\theta}) \\
  &= 1 - \frac{1}{\Ncis}\sum_{i=1}^{\Ncis} \prob(O_j = \varnothing \mid i_j = i, \vec{\theta}) \\
  &= \frac{1}{\Ncis} \sum_{i=1}^{\Ncis} (1 - \prob(O_j = \varnothing \mid i_j = i, \vec{\theta}))
\end{align}

\section{Derivation of quantities in \cref{sec:false-negatives}}

\subsection{Expressions in \cref{imperf-test:sec:modifying-p_ia}} \label{sec:p-ia-dash}

We proceed by first considering whether $E_j > r_k^{(e)}$ is the case and conditioning on $B_j = b$.
Then, we combine the cases and remove the conditioning.

First, the case when $E_j \leq r_k^{(e)}$.
In this case, the test at $r_k^{(e)}+1$ is a true negative and the end of the episode is interval censored as in the previous chapter.
% In this case, the test at $r_k^{(e)} + 1$ is a true negative, as are all other tests not in $\sched'_{k}$.
The true negative occurs with probability 1, by the assumption of no false positives.
\begin{align}
&\prob(O'_j = \vec{\nu}_k', E_j \leq r_k^{(e)} \mid B_j = b, i_j = i_k, \psens, \vec{\theta}) \\
&= \prob(O'_j = \vec{\nu}_k', l_k^{(e)} \leq E_j \leq r_k^{(e)} \mid B_j = b, i_j = i_k, \psens, \vec{\theta}) &\text{the test at $l_k^{(e)}$ is positive} \\
&= \prob(O'_j = \vec{\nu}_k' \mid l_k^{(e)} \leq E_j \leq r_k^{(e)}, B_j = b, i_j = i_k, \psens, \vec{\theta}) \\
&\ \ \  \times \prob(l_k^{(e)} \leq E_j \leq r_k^{(e)} \mid B_j = b, i_j = i_k, \psens, \vec{\theta}) \\
&= p_\text{sens}^{t_+} (1 - p_\text{sens})^{f_-} \left( S_{\vec{\theta}}(l_k^{(e)} - b + 1) - S_{\vec{\theta}}(r_k^{(e)} - b + 2) \right)
\label{imperf-test:eq:ll-ei-lt-ri}
\end{align}

Second, the case when $E_j > r_k^{(e)}$.
In this case, the test at $r_k^{(e)}+1$ is a false negative, occurring with probability $(1 - p_\text{sens})$.
To avoid having to consider tests after $r_k^{(e)}$, which could greatly complicate the likelihood, we model this case as the episode being right censored at $r_k^{(e)}$.
Taking the same approach as before:
\begin{align}
&\prob(O'_j = \vec{\nu}_k', E_j > r_k^{(e)} \mid B_j = b, i_j = i_k, \psens, \vec{\theta}) \\
&= \prob(O'_j = \vec{\nu}_k' \mid E_j > r_k^{(e)}, B_j = b, i_j = i_k, \psens, \vec{\theta}) \\
  &\ \ \  \times \prob(E_j > r_k^{(e)} \mid B_j = b, i_j = i_k, \psens, \vec{\theta}) \\
&= p_\text{sens}^{t_+} (1 - p_\text{sens})^{f_-} (1 - p_\text{sens}) S_{\vec{\theta}}(r_k^{(e)} - b + 2)
\label{imperf-test:eq:ll-ei-gt-ri}
\end{align}

These expressions can now be used to derive $p'_{ik}$.
First, augment the data with $b$, and split into the cases just discussed, omitting the conditioning on $\psens$, $\vec{\theta}$, and $i_j = i_k$:
\begin{align}
p_{ik}'
=& \prob(O'_j = \vec{\nu}_k') \\
=& \sum_{b = l_k^{(b)}}^{r_k^{(b)}} \left( \prob(O'_j = \vec{\nu}_k', E_j \leq r_k^{(e)} \mid B_j = b) + \prob(O'_j = \vec{\nu}_k, E_j > r_k^{(e)} \mid B_j = b) \right) \prob(B_j = b). \\
\intertext{Now, substitute in \cref{imperf-test:eq:ll-ei-lt-ri,imperf-test:eq:ll-ei-gt-ri} and take out the common factor:}
=\ &  p_\text{sens}^{t_+} (1 - p_\text{sens})^{f_-} \\
 & \times \sum_{b = l_k^{(b)}}^{r_k^{(b)}} \left( S_{\vec{\theta}}(l_k^{(e)} - b + 1) - S_{\vec{\theta}}(r_k^{(e)} - b + 2) + (1 - p_\text{sens}) S_{\vec{\theta}}(r_k^{(e)} - b + 2) \right) \\ 
  & \times \prob(B_j = b \mid p_\text{sens}, \vec{\theta}) \\
=\ &  p_\text{sens}^{t_+} (1 - p_\text{sens})^{f_-} \\
  & \times \sum_{b = l_k^{(b)}}^{r_k^{(b)}} \left( S_{\vec{\theta}}(l_k^{(e)} - b + 1) - p_\text{sens} S_{\vec{\theta}}(r_k^{(e)} - b + 2) \right) \\
  & \times \prob(B_j = b \mid p_\text{sens}, \vec{\theta}).
\label{imperf-test:eq:pia-prime}
\end{align}
Note that if $p_\text{sens} = 1$ then $p_{ik}' = p_{ik}$ (see \cref{perf-test:eq:pia}).

We use a fixed $\psens$ (\ie a point prior) and $\prob(B_j = b \mid \psens, \vec{\theta}) \propto 1$ giving:
\begin{align}
p_{ik}'
&\propto \sum_{b = l_k^{(b)}}^{r_k^{(b)}} S_{\vec{\theta}}(l_k^{(e)} - b + 1) - p_\text{sens} S_{\vec{\theta}}(r_k^{(e)} - b + 2).
\end{align}
Estimating $\psens$ is not possible in the current framework (see \cref{sec:discussion}).

\subsection{Expressions in \cref{imperf-test:sec:modifying-p_iu}} \label{sec:p-iu-dash}

The probability of one of the conditions that cause an episode to be missed (specified in \cref{imperf-test:sec:modifying-p_iu}) occurring, conditional on $B_j = b$ where $\min(\sched_{i_j}) < b \leq T_{i_j}$ is:
\begin{align}
&\prob \left(
    \tau_{\sched_{i_j}}(b) + 1 \leq D_j \leq \tau^2_{\sched_{i_j}}(b)
    \mid B_j = b, \vec{\theta} \right) (1 - \psens) \\
&= \left( S_{\vec{\theta}}(\tau_{\sched_{i_j}}(b) + 1) - S_{\vec{\theta}}(\tau^2_{\sched_{i_j}}(b) + 1) \right) (1 - \psens).
\end{align}
Summing over $b$, in the same way as \cref{perf-test:eq:piu}, gives:
\begin{align}
\zeta = (1 - p_\text{sens})\frac{1}{T} \sum_{b=\min(\sched_{i_j}) + 1}^{T_{i_j}} \left( S_{\vec{\theta}}(\tau_{\sched_{i_j}}(b) + 1) - S_{\vec{\theta}}(\tau^2_{\sched_{i_j}}(b) + 1) \right).
\label{imperf-test:eq:zeta}
\end{align}

$p_{iu}'$ is the probability of episode $i$ being undetected, considering both the previous and new mechanisms.
The previous and new mechanisms are mutually exclusive.
Hence, $p_{iu}'$ is the sum of these, $p_{iu}' = p_{iu} + \zeta$.
As previously, $1 - p_{iu}'$ is the required quantity.
\begin{align}
1 - p_{iu}'
=& 1 - p_{iu} - \zeta \\
% =& \frac{1}{T} \sum_{b = \min \sched_{i} + 1}^{T_{i_j}} S_{\vec{\theta}}(\tau_{\sched_{i}}(b) + 1) \\
% &- (1 - p_\text{sens})\frac{1}{T} \sum_{b=\min(\sched_{i_j}) + 1}^{T_{i_j}} \left( S_{\vec{\theta}}(\tau_{\sched_{i_j}}(b) + 1) - S_{\vec{\theta}}(\tau^2_{\sched_{i_j}}(b) + 1) \right) \\
=& \frac{1}{T} \sum_{b=\min(\sched_{i_j}) + 1}^{T_{i_j}} \left( p_\text{sens} S_{\vec{\theta}}(\tau_{\sched_{i_j}}(b) + 1) + (1 - p_\text{sens}) S_{\vec{\theta}}(\tau^2_{\sched_{i_j}}(b) + 1)\right).
\end{align}

\section{ATACCC-based prior} \label{sec:ataccc-prior}

Reliable estimates of $\lambda_t$ for $t$ up to around 20 are available from a prior analysis of data from The Assessment of Transmission and Contagiousness of COVID-19 in Contacts (ATACCC) study~\citep{hakkiOnset}, which tested individuals who had been exposed to infection daily up to a maximum of 20 days.
The infrequent testing in CIS means that there is a lack of information about short infection episodes, and hence we use these estimates as informative priors.

When constructing the prior, two aspects need consideration.
Firstly, the model structure from the ATACCC-based analysis leads to its posterior distribution having a positive correlation between $\lambda_t$ and $\lambda_{t'}$, especially for small $|t-t'|$.
The prior used in this analysis should preserve this correlation.
Secondly, the uncertainty in the prior estimates for $\lambda_t, t\geq20$ are underestimated because they are based on extrapolation of the ATACCC data using strong model assumptions.

We first approximate the previous posterior estimate of the hazard as $\logit{\vec{h}} \dist \MNorm(\vec{\mu}_A, \matr{\Sigma}_A)$ where $\vec{h}$ is the hazard, and $\vec{\mu}_A$ and $\matr{\Sigma}_A$ are the mean and covariance matrix estimated using samples of $\logit{\vec{h}}$ from the previous study's posterior.
Using a multivariate normal, as opposed to multiple univariate distribution for each $h_t$, preserves the correlation between the hazards.
% The logit transformation maps from the $[0, 1]$ interval to the full real line.
The approximation is very good (not shown).

Having approximated the estimate as a multivariate normal, we add additional uncertainty using a discrete Beta process.
The discrete Beta process prior~\citep{ibrahimBayesian,sunStatisticala} generalises the form of prior used in the weakly informative case by allowing the central estimate of the hazard to vary over time.
It is:
\begin{align}
  \logit \vec{h} &\dist \MNorm(\vec{\mu}_A, \matr{\Sigma}_A) \\
  \lambda_t &\dist \text{Beta}(\alpha_t, \beta_t) &t = 1, 2, \dots \\
  \alpha_t &= k_t h_t + \alpha_0 \\
  \beta_t &= k_t (1 - h_t) + \beta_0
\end{align}
where $k_t$, $\alpha_0$, and $\beta_0$ are hyperparameters.
An intuition for what this distribution represents derives from a conjugate model for $\lambda_t$ with a beta prior and a binomial likelihood.
If $\lambda_t$ is given the prior distribution $\text{Beta}(\alpha_0, \beta_0)$, and we then have $k_t$ observations with $k_t h_t$ successes, then the posterior distribution for $\lambda_t$ is $\text{Beta}(\alpha_t, \beta_t)$ (as defined above).

$k_t$ reflects the subjective belief that $h_t$ is a good estimate of $\lambda_t$ for small $t$ but increasingly unreliable.
\begin{align}
k_t = \begin{cases}
  \expit(-0.4 * (t - 20)) &\text{for $t \leq 39$} \\
  0 &t > 39
\end{cases}.
\end{align}

\section{Distribution of duration used in simulation} \label{sec:simulation-truth}

The simulation requires a distribution of the duration of detectability.
We modify the ATACCC-based duration estimate from \citet[chapter 4]{blakeThesis} with an inflated tail to be consistent with the CIS.
The tail inflation uses a simple survival analysis and the CIS data.

This analysis assumes the initiating event is known, and equal to the episode’s detection time, $j_j^{(b)}$.
It assumes the final event is interval censored between the time of the final positive test and the subsequent negative test, or right censored if a negative test has not yet been observed.
A flexible, spline-based form is used for the baseline survival function~\citep{roystonSTPM,roystonFlexible} with covariates introduced via proportional odds.
By not accounting for either the undetected infections or the interval censoring of the initiating event, this analysis has competing biases which makes them hard to interpret~\citep{cisMethodsONS}.

To form the duration distribution used in the simulation, we combine the two estimates.
The pdf over the first 30 days is proportional to the ATACCC estimate, with the rest proportional to this CIS-based estimate.
Denote by $f_A(t)$ the ATACCC-based distribution function and $f_C(t)$ that from the CIS-based estimates just derived.
Then define:
$$
f_S'(t) = \begin{cases}
	f_A(t) &t \leq 30 \\
	f_C(t) &t > 30
\end{cases}
$$
Then the distribution used in the simulation is the normalised version of this: $f_S(t) = f'_S(t)/\sum_i f_S'(i)$.
% These curves and the combined curve are compared in \cref{perf-test:fig:duration-dist}.
Episode $j$'s duration of detectability is then a draw from this distribution.

\bibliographystyle{agsm}

\bibliography{references}

\end{document}