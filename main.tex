\documentclass[12pt]{article}
\usepackage{amsmath}
\usepackage{graphicx}
\usepackage{enumerate}
\usepackage{natbib}
\usepackage{url} % not crucial - just used below for the URL 

%\pdfminorversion=4
% NOTE: To produce blinded version, replace "0" with "1" below.
\newcommand{\blind}{1}
\newcommand{\fulltitle}{Estimating the duration of RT-PCR positivity for SARS-CoV-2 from doubly interval censored data with undetected infections}

% DON'T change margins - should be 1 inch all around.
\addtolength{\oddsidemargin}{-.5in}%
\addtolength{\evensidemargin}{-1in}%
\addtolength{\textwidth}{1in}%
\addtolength{\textheight}{1.7in}%
\addtolength{\topmargin}{-1in}%

%% ABOVE THIS LINE IS JASA TEMPLATE
%% BELOW THIS LINE IS MY STUFF

% add bibliography to TOC - remove for final version
\usepackage[nottoc,numbib]{tocbibind}

\usepackage[utf8]{inputenc}
\usepackage{amssymb}
\usepackage{floatpag}
\usepackage{bm}
% \usepackage{algorithm2e}
\usepackage[unicode,psdextra]{hyperref}
\usepackage[nameinlink]{cleveref}
\usepackage{csquotes}
\usepackage[T1]{fontenc}
\usepackage{textcomp} % provide symbols
\usepackage{xr}
\usepackage{afterpage}
\usepackage{caption}
\usepackage{numprint}
\npfourdigitnosep
\npdecimalsign{.}
% \usepackage{microtype}
\usepackage{todonotes}

% Generic maths commands
\def\reals{\mathbb{R}}
\def\nats{\mathbb{N}}
\def\sampSpace{\mathcal{X}}
\def\dist{\sim}
\DeclareMathOperator{\E}{\mathbb{E}}
\DeclareMathOperator{\V}{\mathbb{V}}
\DeclareMathOperator{\I}{\mathbb{I}}
\DeclareMathOperator{\prob}{\mathbb{P}}
\DeclareMathOperator{\p}{\pi}
\DeclareMathOperator{\var}{\mathbb{V}}
\DeclareMathOperator{\indicator}{\mathbb{I}}
\DeclareMathOperator{\cov}{Cov}
\DeclareMathOperator{\cor}{Cor}
\DeclareMathOperator{\logit}{logit}
\DeclareMathOperator{\Ber}{Bernoulli}
\DeclareMathOperator{\Bin}{Binomial}
\DeclareMathOperator{\Poi}{Poisson}
\DeclareMathOperator{\BetaDist}{Beta}
\DeclareMathOperator{\Exponential}{Exponential}
\DeclareMathOperator{\NBr}{NegBin}
\newcommand{\NBc}{\NBr_{c}}
\newcommand{\NBs}{\NBr_{s}}
\DeclareMathOperator{\BB}{BetaBin}
\DeclareMathOperator{\GamDist}{Gamma}
\DeclareMathOperator{\MN}{Multinomial}
\DeclareMathOperator{\N}{N}
\DeclareMathOperator{\MNorm}{N}
\DeclareMathOperator{\LN}{LN}
\DeclareMathOperator{\LKJ}{LKJ}
\DeclareMathOperator{\expit}{expit}
\newcommand\matr{\bm}
\newcommand\set{\mathcal}
\renewcommand{\vec}[1]{\bm{#1}}
\newcommand{\ssep}{:}
\DeclareMathOperator*{\argmax}{arg\,max}

\newcommand\citePersonalComms[1]{(#1, personal communication)}

% Thesis-specific maths commands
\newcommand{\dmax}{d_\text{max}}
\newcommand{\psens}{p_\text{sens}}
\newcommand{\psenss}{p_\text{sens}^{(s)}}
\newcommand{\psensi}{p_\text{sens}^{(i)}}
\newcommand{\ntot}{n_\text{tot}}
\newcommand{\ndet}{n_\text{d}}
\newcommand{\nnodet}{n_\text{u}}
\newcommand{\pnodet}{p_\text{u}}
\newcommand{\Npop}{N_\text{pop}}
\newcommand{\Ncis}{N_\text{CIS}}
\newcommand{\ncis}{\vec{n_\text{CIS}}}
\newcommand{\na}{\vec{n}_\text{obs}}
\newcommand{\pcis}{\vec{p_\text{CIS}}}
\newcommand{\sched}{\mathbb{T}}
\newcommand{\nsched}{n_{\sched}}
\newcommand{\inform}{{_{\text{inform}}}}


%% Bibliography
% \usepackage[authordate-trad,backend=biber]{biblatex-chicago}
% \addbibresource{references.bib}

% Macros for common abbreviations to get the spacing right
% See: https://stackoverflow.com/questions/3282319/correct-way-to-define-macros-etc-ie-in-latex
\usepackage{xspace}
\makeatletter
\DeclareRobustCommand\onedot{\futurelet\@let@token\@onedot}
\def\@onedot{\ifx\@let@token.\else.\null\fi\xspace}
\def\eg{e.g\onedot} \def\Eg{{E.g}\onedot}
\def\ie{i.e\onedot} \def\Ie{{I.e}\onedot}
\def\cf{c.f\onedot} \def\Cf{{C.f}\onedot}
\def\etc{etc\onedot} \def\vs{{vs}\onedot}
\def\wrt{w.r.t\onedot} \def\dof{d.o.f\onedot}
\def\etal{et al\onedot}
\makeatother

%% LINE NUMBERS
% \usepackage{lineno} % Include the package for line numbering
% \linenumbers % Activates line numbering for the document

\begin{document}

%\bibliographystyle{natbib}

\def\spacingset#1{\renewcommand{\baselinestretch}%
{#1}\small\normalsize} \spacingset{1}


%%%%%%%%%%%%%%%%%%%%%%%%%%%%%%%%%%%%%%%%%%%%%%%%%%%%%%%%%%%%%%%%%%%%%%%%%%%%%%

\if1\blind
{
  \title{\bf \fulltitle}
  \author{%
    Blake, TBC, Birrell, De Angelis  
    % Author 1\thanks{
    % The authors gratefully acknowledge \textit{please remember to list all relevant funding sources in the unblinded version}}\hspace{.2cm}\\
    % Department of YYY, University of XXX\\
    % and \\
    % Author 2 \\
    % Department of ZZZ, University of WWW}
  }
  \maketitle
} \fi

\if0\blind
{
  \bigskip
  \bigskip
  \bigskip
  \begin{center}
    {\LARGE\bf \fulltitle}
\end{center}
  \medskip
} \fi

\bigskip
\begin{abstract}
The text of your abstract. 200 or fewer words.
\end{abstract}

\noindent%
{\it Keywords:}  3 to 6 keywords, that do not appear in the title
\vfill

\newpage

\listoftodos

The \textbf{format requirements from JASA} are at \url{https://www.tandfonline.com/journals/uasa20}.
\begin{itemize}
    \item Should be written with the following elements in the following order: title page; author footnote; abstract; keywords; article text (table(s); figures); acknowledgments; appendices; references
    \item Should be no more than 35 pages, including the abstract, figures, tables, and references. Appendices, proofs, and other supporting material should be placed in a separate supplement file (anonymized for review)
    \item Should contain an unstructured abstract of 200 words.
    \item Should contain between 3 and 5 keywords. Read making your article more discoverable, including information on choosing a title and search engine optimization.
    \item JASA requires that all papers be formatted for 8 1/2 x 11-inch paper, one side only.
    \item 12 point, double-spaced font (which we define as 26 lines of text per page)
    \item the average JASA paper is about 30 manuscript pages (with the above page format and including all appendices, references, tables, and figures), and it is very uncommon for papers much longer than about 35 pages to appear in JASA. 
\end{itemize}

\tableofcontents

\newpage
\spacingset{1.9} % DON'T change the spacing!
\section{Introduction} \label{sec:intro}

The pandemic caused by the SARS-CoV-2 virus is estimated to have killed approximately 30 million people globally~\citep{whoCOVIDExcess,economistCOVIDExcess}, stretched healthcare services to the brink of collapse~\citep{fongNHS}, and disrupted societies worldwide.
This pandemic highlighted the critical need for an effective public health response, informed by quantifying both the current and future burden, including the incidence of infection, \ie the rate at which new infections occur.
This metric is rarely observed directly and thus requires estimation.


% The unprecedented scale of this motivated large-scale surveillance studies of a type never before seen.
% The analysis of these studies pose novel statistical questions, one of which we explore here.
% These included large-scale prevalence surveys, which estimate the proportion of the population currently positive, combined with back-calculation methodology. HOW DOES IT WORK?

Incidence of infection can be estimated through back-calculation~\citep{brookmeyerBackcalculation}, a methodology which relates observations of a disease outcome with its incidence through a delay distribution and a convolution equation.
In the context of SARS-CoV-2, this approach allows estimation of incidence based on: a time series of the proportion of the population that have a detectable infection; and the distribution of the duration of infection episodes, \ie the duration of time an individual is detectable for following infection.
As SARS-CoV-2 is detected by performing RT-PCR testing on an appropriate sample (\eg a nasal swab), this duration is the period over which an infected individual would return a positive RT-PCR result.
In the UK, estimates of the proportion of the population that is detectable are available from to large-scale prevalence surveys~\citep{cisMethodsONS,rileyREACT}.
Here, we estimate the second back-calculation component: the distribution of the duration of infection episodes.
% Back-calculation requires knowledge of the distribution of the length of time  an individual remains positive after acquiring an infection.
% In the context of SARS-CoV-2, and other RNA viruses positivity is measured by performing RT-PCR testing on an appropriate sample (\eg a nasal swab) and its duration is the period during which an infected individual would return a positive RT-PCR result. 


% RT-PCR testing is a highly sensitive and specific technique to detect the presence of a specific RNA sequence in a sample.
% Therefore, the required distribution is the duration of positivity for SARS-CoV-2 RT-PCR testing.

An early meta-analysis estimated the mean duration as 14.6 days (95\% CI: 9.3--20.0 days)~\citep{cevikShedding}; however, most studies included in the meta-analysis involved short follow-up and/or hospitalised patients or had unclear inclusion criteria, and hence may not be representative of the general population.
This remains true of later studies~\citep[e.g.][]{ealesCharacterising,hellewellPCRSensitivity}.
% Furthermore, studies tend to have short follow-up and may not provide good estimates of the mean duration of positivity, which is a right-skewed distribution.
% For backcalculation, the mean is the most important feature~\citep{freemanPrevalence}, but results can be sensitive to the entire distribution.

Other studies have estimated the duration by inferring the dynamics of viral load, \ie the quantity of virus in an individual~\citep[e.g.][]{kisslerViral,hakkiOnset,blakeThesis}.
The work of \citet{hakkiOnset} is particularly relevant, as it is based on a representative sample of individuals exposed to infection and subsequently tested daily.
However, the follow-up lasted for a maximum of 20 days, preventing estimation of the upper tail of the duration distribution.
This issue is exacerbated by the small sample size (\ie $n= 57$ infections).

In this paper, we investigate an alternative source of data for the estimation of the duration of an infection episode, the Coronavirus Infection Survey (CIS)~\citep{CIS}.
The CIS is a unique large-scale longitudinal study of RT-PCR positivity in the general population, with around 500,000 individuals followed at regular intervals for a long period of time (see \cref{sec:data}).
% The size of the cohort varied, at the period we are concerned with is \numprint{437590}.
%% We will consider \numprint{4800} infections.
As the testing schedule was independent of infection status and had indefinite length, the study provides a unique opportunity to provide an unbiased estimate of the distribution of infection episode duration in the general population.
However, the CIS study design, with weekly and four-weekly testing intervals, and its implementation pose several challenges to the estimation.

Firstly, infections can be undetected.
Four-weekly testing is longer than many infection episodes, so that they could begin and end between tests.
Specifically, for individuals without a positive test, we do not know if they were infected but undetected, or if they were never infected (compare the two outcomes in \cref{fig:challenges}(A)).
This is different from standard left truncation~\cite[e.g.][]{sunEmpirical,bacchettiNonparametric} because we have information on the individuals in which the short infection episodes occur (\ie their testing times), which constrains undetected infection episode lengths.
% This is subtly different from the standard left truncation problem~\cite[e.g.][]{sunEmpirical,bacchettiNonparametric} where individuals that have experienced a terminating event are excluded from the study and no information is available on them. In CIS, individuals in which undetected infections occur are enrolled in the study and their testing times are known, providing bounds on how long any possible infection episodes may have lasted.
% We refer to these events as ``undetected'', rather than truncated, to emphasise this difference.

Secondly, we observe the beginning of an episode when a previously negative individual returns a positive test.
However, the episode could start at any point between those two tests; \ie it is interval censored (see \cref{fig:challenges}(B)).
The end of the episode is similarly interval censored and hence the data is doubly interval censored.

\begin{figure}
  \makebox[\textwidth][c]{\includegraphics{figures/output/challenges}}
  \caption{%
    Challenges posed by the CIS design.
    (A) Undetected episodes.
    (B) Doubly interval censored episodes.
  }
  \label{fig:challenges}
\end{figure}


Thirdly, test results can be misclassified.
The sensitivity, although not specificity, of RT-PCR tests is substantially less than 100\%~\citep{cisMethodsONS}.
In particular, the negative test, providing the upper bound to an episode could be a false negative, incorrectly resulting in a shorter duration.  Furthermore, this increases the number of episodes undetected.

Fourthly, there is a lack of information on the first 13 days of the duration distribution, which it is important to characterise correctly.
The most precise information on an infection episode lasting 13 days is provided by a single positive test with negative tests in the preceding and following seven days.
Data from the CIS dataset does not allow to distinguish whether this infections episode lasted one day, 13, or anywhere in-between.
% Furthermore, short infection episodes are much more likely to be undetected.
% Therefore, there is a lack of information on what proportion of episodes are short.

A further challenge is posed by the fact that the CIS data constitute a highly sensitive dataset and is stored on a Trusted Research Environment, the ONS Secure Research Service (SRS)~\citep{onsSRS}.
The SRS possesses limited computational power which has to be shared across multiple users, resulting in little capacity for computationally intensive methods.
% Specifically, it contains five \numprint[GHz]{2.40} dual-core Intel Xeon E5-4650 processors and \numprint[GB]{164} RAM.


% \subsection{Related work}

Methods that deal with doubly interval censored data are available~\citep{sunStatistical,bogaertsSurvival}.
However, few consider the additional challenges posed by undetected episodes with
theoretical frameworks~\citep{turnbullEmpirical,dempsterMaximum} only applied to the special case where the terminating event is either uncensored or right censored~\citep[e.g.][]{sunEmpirical,bacchettiNonparametric,shenNonparametric}.
Inference in these studies depends on the interval lengths being fairly consistent throughout.
% In these studies inference is based on a likelihood contribution for each detected infection episode conditional on the interval in which the episode is known to begin. For this conditional likelihood to be valid, it is assumed that this conditioning provides negligible information on the distribution of interest~\citePersonalComms{Nick Jewell}.
% This assumption is reasonable when the length of the intervals is fairly constant.
% In particular, when the interval lengths are near-constant, the probability of detecting an episode is independent of when it occurs.
However, the interval lengths change from one to four weeks in the CIS. %episodes are more likely to be detected when an individual is tested weekly, in their first four weeks of testing.
% Conditioning on the interval in which the episode begins is mathematically convenient; however, it will lead to an overestimation of the probability of detection.
% Infection episodes in an individual's one-week testing period are over-represented in the detected infections.
% This is because more frequent testing will detect a higher proportion of infections.
% Therefore, $p(\text{episode $j$ detected} \mid b_j \in [l_j^{(b)}, r_j^{(b)}]) > p(\text{episode $j$ detected})$ and conditioning on $b_j \in [l_j^{(b)}, r_j^{(b)}]$ is inappropriate.

\citet{heiseyModelling} generalise the approaches above to allow arbitrary patterns of detection times and censoring.
They categorise each combination of possible beginning and end times for an episode into whether the episode would be detected, and whether it is compatible with the pattern of interval censoring observed.
This allows them to build a conditional likelihood, accounting for both these challenges in a context where possible detection times are common to all individuals, leading to a simplification of the likelihood and a common probability of detection across all individuals.
%Their approach was developed in the context of assessing the survival time of bird and insect nests, and has since been used widely~\citep{heiseyABCs}.
% However, in this context, the visit times (possible detection times) are common to all the individuals, simplifying the likelihood and the structure of the missing data.
% In particular, the probability of detection is common across all individuals.


The inclusion of false negatives into survival analysis is an area of much interest, with
\citet{piresIntervalMisclassify} providing a comprehensive review of approaches. However, including false negatives with either doubly interval censored data or missed events have not been addressed. The CIS data requires addressing both.

In what follows, we adopt a Bayesian survival analysis framework to estimate the distribution of infection episode durations from the CIS data.
A Bayesian framework allows incoporation of prior knowledge to address the lack of information over short episodes~\citep{caoBias}; uncertainty quantification, which is otherwise challenging in this setting~\citep{sunStatistical,dengNonparametric}; and it provides the ability to include the undetected infections through data augmentation.

The paper is structured as follows: the survival analysis framework is developed in \cref{sec:inference}, after we describe the design of the CIS and the data it collects in \cref{sec:data}.
In \cref{sec:false-negatives}, we extend the framework to incorporate false negative tests and in \cref{sec:parameters-priors} propose two possible priors for the survival time.
We test the framework in a simulation study in \cref{sec:simulation}, then apply it to the CIS data in \cref{sec:CIS}.
We conclude with a discussion in \cref{sec:discussion}.


\section{Study design and data} \label{sec:data}

The CIS~\citep{CIS} was set up in April 2020.
The dataset is globally unique in providing a representative, longitudinal, and large-scale study across the pandemic.
Individuals were visited on a pre-determined schedule.
At each visit, a questionnaire was completed and a combined throat and nose swab was collected for RT-PCR testing; a subset of individuals and visits also collected a blood sample for serological testing.
We consider only the result of the RT-PCR testing.
% Initially, it was limited to England, but expanded to cover the whole of the UK in September 2020.
Enrolment was continuously ongoing until 31st Jan 2022, with data collected until 13th Mar 2023~\citep{weiRisk}. 

We focus on the period between 10 Oct 2020 and 6 Dec 2020 inclusive over which the CIS estimated stable infection prevalence of around 1\%~\citep{onsCISdec2020}.
Additionally, it is prior to either vaccination or the emergence of the alpha variant, which potentially impacted the duration of infection episodes~\citep{hakkiOnset,russellWithinhost}.
Our analysis considers a cohort of $\Ncis = \numprint{437590}$ individuals with at least one test in the chosen period.

The CIS had a household-based design inviting all individuals aged over 2 years from selected households.
Once invited, an enrolment swab would be taken at the first visit followed by 4 further weekly visits (giving a total of 5 swabs on days 0, 7, 14, 21, 28 relative to enrolment) after which visits were four-weekly.
The gap between visits means that infection episodes can be undetected, in which case the observations are indistinguishable from an individual remaining undetected (see \cref{fig:challenges}(A)).
In reality, visits were often not on this precise schedule, and occasionally visits were missed %%  (\eg a test would occur on days 28 and 84 but not 56).
A full description of the study can be found in the study protocol~\citep{cisProtocol}.

We denote by $\sched_i$ the set of times individual $i = 1, \dots, \Ncis$ is tested.
Time is defined such that the first day of the period considered, 10 Oct 2020, is day 1 and the final day of the period considered, 6 Dec 2020, is day $T$.
The smallest element of $\sched_i$ is $i$'s last test prior to time 1, if it exists, or their first test following enrolment in the study otherwise, and including all subsequent tests even those that occur after day $T$.
Each individual has exactly one test schedule, but it is possible that $\sched_i = \sched_{i'}$ for $i \neq i'$.
We assume that the test schedules are uninformative on all quantities of interest, in particular the presence or absence of infection, because they are part of the study design.
Therefore, we condition on them implicitly in all the calculations that follow.


% Positive tests from the same individual are grouped into infection episodes.
Positive tests with negatives in-between may or may not be classed as the same infection episode based on the time between the tests, the number of negative tests between the positives, and the variant of the infection; see \citet{weiRisk} for details.
% Negative tests between positives in the same episode are known as \emph{intermittent negatives}; these are false negative tests.
% Other tests could also be misclassified, but this is not known.

The crucial features of an infection episode are the start and end times of the infection.
Ignoring misclassified test results, the latent time at which infection episode $j$ begins, $B_j$, is bounded between, $l_j^{(b)}$, the day after the last negative test, and $r_j^{(b)}$, the day of the first positive test (see \cref{fig:challenges}(B)); \ie $B_j \in [l_j^{(b)}, r_j^{(b)}]$.
Similarly, $E_j$, the latent end time of infection episode $j$ and is bounded by $l_j^{(e)}$, the day of the last positive test and $r_j^{(e)}$, the day before the following negative test; \ie $E_j \in [l_j^{(e)}, r_j^{(e)}]$.
We define the infection episode such that the individual is detectable between days $B_j$ and $E_j$ inclusive.
Therefore, all the relevant information about a detected infection episode $j$ is fully contained in the vector $o_j = [l_j^{(b)}, r_j^{(b)}, l_j^{(e)}, r_j^{(e)}, i_j]^T$ where $i_j$ is the individual in which episode $j$ occurs.

The duration of infection episode $j$, $D_j$, is the number of days on which a positive result would be returned by a RT-PCR testing procedure that has 100\% specificity and sensitivity.
Hence, $D_j = E_j - B_j + 1$.
The distribution is commonly expressed using the survival function, $S_{\vec{\theta}}(d) = \prob(D_j \geq d \mid \vec{\theta})$, and is characterised by parameter vector $\vec{\theta}$.
We assume that the $D_j$s are independent of each other and $B_j$, and identically distributed.

%We analyse the $\ndet = \numprint{4800}$
We include only episodes for which: (i) the episode's first positive test occurred between 10 Oct 2020 and 6 Dec 2020 inclusive, \ie $r_j^{(b)} \in [1, T]$; and (ii) a negative test bounds both the beginning and end of the episode,
therefore, both $l_j^{(b)}$ and $r_j^{(e)}$ exist.
In total, there were $\ndet = \numprint{4800}$ such detected episodes.

\section{Inference framework} \label{sec:inference}

The target of inference is $\vec{\theta}$, the parameters of the survival function $S_{\vec{\theta}}(t) = \prob(D_j \geq t \mid \vec\theta)$ for random variable $D_j$, the duration of the $j$th infection episode.
We adopt a Bayesian framework to derive a posterior distribution for $\vec{\theta}$ given the partial information provided by the set of observed $o_j$ vectors and accounting for the double interval censoring and undetected episodes.
The composition of $\vec{\theta}$ and appropriate priors for $\theta$ are discussed in \cref{sec:parameters-priors}.
In this section, we derive a statistical model for the case where that there are no misclassified test results; in \cref{sec:false-negatives} we generalise the framework to include false negatives.

We assume the observation $o_j = [l_j^{(b)}, r_j^{(b)}, l_j^{(e)}, r_j^{(e)}, i_j]^T$ is a realisation of the discrete random variable $O_j$, taking values in $\set{E}' = \{ \emptyset \} \cup \set{E}$ where $\set{E} = \{ \vec{\nu}_1, \dots, \vec{\nu}_{N_E} \}$ is the set of all $N_E$ (a priori) possible unique observations of detected episodes and $o_j = \emptyset$ if and only if $o_j$ is undetected.
Therefore, $\vec{\nu}_k = [l^{(b)}_k, r^{(b)}_k, l^{(e)}_k, r^{(e)}_k, i_k]^T$ for $k = 1, \dots, N_E$.
The state space is illustrated in \cref{perf-test:fig:partitionSpace}.
% Infection episodes are indexed by $j = 1, \dots, \ntot$.
% If $j$ is a detected episode, then $o_j = [l_j^{(b)}, r_j^{(b)}, l_j^{(e)}, r_j^{(e)}, i_j]^T$, as before. 
% If $j$ is undetected, then define $o_j = \emptyset$.

\begin{figure}
    \makebox[\textwidth][c]{\includegraphics[width=0.9\paperwidth]{figures/output/regions_diag}}
    \thisfloatpagestyle{empty}
    \caption[Episode regions]{%
      Each dot is a combination of $b_j$ and $e_j$ for an arbitrary individual $i$.
      The combinations giving rise to the same value of $\nu_k$ are in the same box, bounded by dashed lines.
      $i$ had negative tests at times 0, 7, 14, 56, and 84 (not shown) and positive tests at times 21 and 28.
      The purple region corresponds to a doubly interval censored episode observed in this individual.
      That is, $n_8 = 1$ and $n_k = 0$ for $k = 1, \dots, 7, 9, \dots, 15$.
      The red region corresponds to combinations giving $O_j = \emptyset$.
      The grey impossible region violates $b_j \leq e_j$.
    }
    \label{perf-test:fig:partitionSpace}
\end{figure}

Let $p_k = \prob(O_j = \vec{\nu}_k \mid \vec{\theta})$, the probability that $O_j$ takes the value $\vec{\nu}_k$.
Similarly, let $p_u = \prob(O_j = \emptyset \mid \vec{\theta})$, the probability that $j$ is undetected.
Then, the probability distribution for $O_j$ is specified by $\vec{p} = [p_1, \dots, p_{N_E}, p_u]^T$.

Let $n_k$ denote the number of times $\vec{\nu}_k$ is observed; $\nnodet$ denote the latent number of undetected episodes; and $\vec{n} = [n_1, \dots, n_{N_E}, \nnodet]^T$.
Formally, $n_k = \sum_{j=1}^{\ntot} \indicator(O_j = \vec{\nu}_k)$ and $\nnodet = \sum_{j=1}^{\ntot} \indicator(O_j = \emptyset)$.
Let $\na = [n_{1}, \dots, n_{N_E}]^T$ be the observed portion of $\vec{n}$; $\na$ contains all the relevant information we observe.
The total number of infection episodes is $\ntot = \ndet + \nnodet$.

If $\vec{n}$ was known, then the problem reduces to the well-studied problem of inferring a distribution from doubly interval censored data; however, only $\na$ is observed.
Therefore, to infer $\theta$, we augment $\na$ with the latent quantity $\nnodet$.
Assume, for tractability, that the events $O_j = \vec{\nu}_k$ and $O_{j'} = \vec{\nu}_{k'}$ are independent for $j \neq j'$ (conditional on $\theta$); this assumption is discussed in \cref{sec:discussion}.
Hence:
\begin{align}
  \vec{n} \mid \ntot, \vec{\theta} &\dist \MN(\ntot, \vec{p})
\intertext{that is:}
  p(\vec{n} \mid \ntot, \vec{\theta}) &= \frac{\ntot!}{\nnodet!\prod_{k=1}^{N_D} n_k!} p_u^{\nnodet} \prod_{k=1}^{N_D} p_k^{n_k}.
  \label{perf-test:eq:multinomial-ll}
\end{align}

In the CIS data, each $n_k$ ($k \neq u$) is observed as either 0 or 1.
Define $\set{D} = \{ k \ssep n_k = 1 \}$, the detected episodes.
Furthermore, note that the support of the multinomial distribution requires that $\nnodet = \ntot - \ndet$.
Then \cref{perf-test:eq:multinomial-ll} simplifies to:
\begin{align}
  p(\vec{n} \mid \ntot, \vec{\theta})
  &= p(\na \mid \ntot, \vec{\theta}) \\
  &= \frac{\ntot!}{(\ntot - \ndet)!} p_u^{\ntot-\ndet} \prod_{k \in \set{D}} p_k.
  \label{perf-test:eq:multinomial}
\end{align}

The relevant information from the CIS data is fully contained in the vector $\na$.
Therefore, the posterior of interest is (full derivations of this and subsequent quantities are in \cref{sec:derivations}):
\begin{align}
p(\vec{\theta} \mid \na)
&\propto  p(\vec{\theta}) \left( \prod_{k \in \set{D}} p_k \right) \left( \sum_{\ntot=\ndet}^\infty p(\ntot \mid \vec{\theta}) \frac{\ntot!}{(\ntot - \ndet)!} \pnodet^{\ntot - \ndet} \right).
\label{perf-test:eq:posterior1}
\end{align}

For mathematical convenience, we assume the prior $\ntot \dist \NBc(\mu, r)$, and that it is independent of $\vec{\theta}$, the parameters of the survival distribution.
In this case \cref{perf-test:eq:posterior1} simplifies to:
\begin{align}
p(\vec{\theta} \mid \na)
&\propto p(\vec{\theta}) \left( \prod_{i \in \set{D}} p_k \right) (r + \mu (1- \pnodet))^{-(r+\ndet)} \label{perf-test:eq:full-posterior}.
\end{align}

The rest of this section derives expressions for each of $p_{k}$, $p_{u}$ and $\eta$.

Decompose $p_k$ as $p_k = p_{ik} \prob(i_j = i_k \mid \vec{\theta})$
where $p_{ik} = \prob(O_j = \vec{\nu}_k \mid i_j = i_k, \vec{\theta})$.
% This is valid as $\prob(O_j = \nu_k \mid i_j \neq i_k) = 0$ due to the condition here being equivalent to equating the two vectors' final elements.
Assume that each infection episode occurs independently and with equal probability in any individual, \ie $\prob(i_j = i_k) = 1/\Ncis$ for all $j$ and $k$, and $B_j$ is uniformly distributed over the length of the epidemic.

Therefore, $p_{ik}$ takes the standard form of the likelihood for doubly interval censored data without truncation~\citep[e.g.][]{sunEmpirical}:
\begin{align}
p_{ik}
\propto& \sum_{b = l_k^{(b)}}^{r_k^{(b)}} \left( S_{\vec{\theta}}(l_k^{(e)} - b + 1) - S_{\vec{\theta}}(r_k^{(e)} - b + 2) \right).
\label{perf-test:eq:pia}
\end{align}

The remaining component of \cref{perf-test:eq:full-posterior} required is $1- p_u$, one minus the probability of missing an infection, \ie the probability of detecting an infection.

\subsection{Deriving $1 - p_u$} \label{sec:prob-undetected}

The final component of \cref{perf-test:eq:full-posterior} required is $1 - p_u$.
\Cref{sec:derivations} shows that
\begin{align}
  1 - p_u
  &= \frac{1}{\Ncis} \sum_{i=1}^{\Ncis} (1 - \prob(O_j = \emptyset \mid i_j = i, \vec{\theta})).
  \label{perf-test:eq:pu}
\end{align}
%% assuming $P(i_j = i \mid \vec{\theta}) = 1/\Ncis$ as before.
Let $p_{iu} = \prob(O_j = \emptyset \mid i_j = i, \vec{\theta})$.
%% Therefore, the crucial component is $1 - p_{iu}$.
%% This is one minus the probability that an episode in individual $i$ was undetected, \ie the probability of the episode being detected.
An episode $j$ in individual $i_j$ is detected if and only if all the following conditions are met.
\begin{enumerate}
    \item $\exists t \in \sched_{i_j}$ such that $b_j \leq t \leq e_j$; this condition is equivalent to having at least one positive test for the episode.
    \item $b_j > \min \sched_{i_j}$.
      For individuals enrolled during the period considered ($\min \sched_{i_j} > 0$), this ensures that the beginning of the episode is lower bounded; thereby adjusting for those who enrolled during the period being less likely to have a detected infection.
      For individuals enrolled prior to the period considered ($\min \sched_{i_j} \leq 0$), this means that the episode was not detected prior to time 1.
    \item $b_j \leq T_{i_j}$ where $T_{i_j} = \max \{ t \in \sched_{i_j} \ssep t \leq T \}$ is the last time that $i_j$ is tested in the period, meaning that the test is detected within the period.
    \item $\exists t \in \sched_{i_j}$ such that $t > e_j$, upper bounding the end of the episode.
      For episodes detected in the period we consider, a negligible number of episodes are excluded due to this criteria.
      It only occurs in individuals lost to follow-up; therefore, we neglect this possibility.
      % For a new context, including recent infections, this condition could be relaxed by considering episodes that do not meet this criterion as right censored.
\end{enumerate}

% An episode is undetected if and only if no tests are performed during the episode or if there was no negative test prior to the episode.
% Equivalently, that the first test at or after $b$ is after $e$, or that there is no negative test prior to $b$.
First define $\tau_{\sched_i}(t)$ as the time until the next test at or after time $t$ in the schedule $\sched_i$:
\begin{align}
\tau_{\sched_i}(t) &= \min \{ t' \in \sched_i : t' \geq t \} - t
\label{perf-test:eq:tau-def}
\end{align}
% defining $\min \emptyset = \infty$; \ie $\tau_{\sched_i}(t) = \infty$ if there is no $t' \in \sched_i$ such that $t' \geq t$.
The first condition can now be expressed as $e_j \geq b_j + \tau_{\sched_{i_j}}(b_j)$.
Equivalently, $d_j \geq \tau_{\sched_{i_j}}(b_j) + 1$.
% The fourth condition can be expressed as $\tau_{\sched_{i_j}}(b_j) < \infty$.


% Then $\Omega_i$ can be written as:
% \begin{align}
% \Omega_i = \{ (b, e) \ssep \tau_{\sched_i}(b) + b > e \vee b \leq \min(\sched_i) \}.
% \end{align}
Therefore, omitting the conditioning on $\vec{\theta}$ and $i_j = i$:
\begin{align}
1 - p_{iu}
% &= 1 - \prob((B_{j}, E_{j}) \in \Omega_i) \\
&= \prob(D_j \geq \tau_{\sched_{i}}(B_j)+ 1, \min \sched_{i} < B_j \leq T_{i}) \\
&= \sum_{b = \min \sched_{i} + 1}^{T_{i}} \prob(D_j \geq \tau_{\sched_{i}}(b) + 1 \mid B_j = b) \prob(B_j = b)\\
&\propto \sum_{b = \min \sched_{i} + 1}^{T_{i}} S_{\vec{\theta}}(\tau_{\sched_{i}}(b) + 1).
\label{perf-test:eq:piu}
\end{align}

Substituting into \cref{perf-test:eq:pu}:
\begin{align}
1 - p_u
& \propto \sum_{i=1}^{\Ncis} \sum_{b = \min \sched_{i} + 1}^{T_{i}} S_{\vec{\theta}}(\tau_{\sched_{i}}(b) + 1).
\end{align}

For computational efficiency, note that this can be rewritten as:
\begin{align}
  1- p_u
  &\propto \sum_{t=1}^{\dmax} S_{\vec\theta}(t) m_t
\intertext{where}
  m_t &= \sum_{i=1}^{\Ncis} \sum_{b = \min \sched_{i} + 1}^{T_{i}} \indicator(\tau_{\sched_{i}}(b) + 1 = t)
\end{align}
where $\indicator$ is the indicator function.
The $m_t$s rely only on the test schedules, which are fixed, and can be computed once and stored.
Furthermore, the sum can be efficiently implemented as a dot product.

\section{Handling false negatives} \label{sec:false-negatives}

Now we modify the survival framework to incorporate false negatives by assuming a test sensitivity, $\psens < 1$.
Using a simple model, in particular a constant test sensitivity, means calculating the likelihood remains tractable.

% We start, in \cref{imperf-test:sec:notation}, by introducing some additional notation needed.
% We then modify the components of the likelihood in \cref{imperf-test:sec:modifying-p_ia,imperf-test:sec:modifying-p_iu} to incorporate false negatives.
Introducing false negatives means that $O_j$, the observation vector of episode $j$, is now random, even if $i_j$, $B_j$, and $E_j$ are known and the results between the tests that bound episode $j$'s beginning and end times enter into the likelihood.
For tractability, we include only tests between the negative tests at $l_j^{(b)}-1$ and $r_j^{(e)}+1$. 
%To formalise this, we define analogous variables to those in \cref{sec:inference}.
%To start with, define the relevant testing times for any episode $j$ with $O_j = \vec{\nu}_k$ as 
Define this set of tests to be $\sched'_k = \{ t \in \sched_{i_k} \ssep r_k^{(b)} \leq t \leq r_k^{(e)} + 1 \}$ and let $m_k$ denote the size of this set.
Denote the elements of $\sched'_k$ by $t_{k,1} < \dots < t_{k,m_k}$.
Let $\set{E}' = \{ \vec{\nu}_1', \dots, \vec{\nu}_{N_E'}' \}$ be a new set of possible observations of detected episodes, including the pattern of positive and negative tests.
Represent the observations such that for any $\vec{\nu}_k' \in \set{E}'$, $\vec{\nu}_k' = [\vec{\nu}_{k}, \vec{y}_k]^T$, where $\vec{y}_k \in \{ 0, 1 \}^{m_k}$ with $y_{k,l}$ being the test result at time $t_{k,l}$.

$\vec{\nu}'_k \in \set{E}'$ occurs if and only if all the following conditions are met.
\begin{enumerate}
  \item $\vec{\nu}_k \in \set{E}$.
  \item The elements of $\vec{y}_k$ corresponding to the tests at times $r_k^{(b)}$ and $l_k^{(e)}$ are positive, \ie $y_{k,1} = y_{k,m_k-1} = 1$.
  \item The element of $\vec{y}_k$ corresponding to the test at time $r_k^{(e)} + 1$ is negative, \ie $y_{k,m_k} = 0$.
\end{enumerate}
The final two conditions are due to the construction of the intervals as positive and negative tests bounding the beginning and end times of the episode.
As condition 2 requires $y_{k,1} = 1$ and condition 3 requires $y_{k,m_k} = 0$, these conditions can only occur if $m_k \geq 2$.

Let $O'_j$ be the observations generated by episode $j$ including the new vector of results.
If $j$ is detected then $O'_j = [O_j, Y_j]^T \in \set{E}'$; otherwise, $O'_j = \emptyset$.

Similarly, we define $p_k'$, $p_{ik}'$, $p_u'$, and $p_{iu}'$ to replace $p_k$, $p_{ik}$, $p_u$, and $p_{iu}$ respectively.
\begin{align}
    p_{ik}' &= \prob(O'_j = \vec{\nu}'_k \mid i_j = i_k, \vec{\theta}) \\
    p_k' &= \frac{1}{\Ncis} p_{ik}' \\
    p_{iu}' &= \prob(O'_j = \emptyset \mid i_j = i, \vec{\theta}) \\
    p_u' &= \frac{1}{\Ncis} \sum_{i=1}^{\Ncis} p_{iu}'.
\end{align}

\subsection{Deriving $p'_{ik}$} \label{imperf-test:sec:modifying-p_ia}

We will modify $p_{ik}$ to form $p_{ik}' = \prob(O'_j = \vec{\nu}'_k \mid i_j = i_k, \vec{\theta})$, taking into account false negatives.
We consider a mixture of two scenarios, defined by whether the final test in $\sched'_{k}$ is a false negative, with the mixture probability determined by the test sensitivity.

Similar likelihoods have previously appeared in the literature~\citep[e.g.][eq.\ (2)]{piresIntervalMisclassify}.
However, this prior work was for singly interval censored data; incorporating the double interval censored nature of the CIS data involves summing over the possible episode start times.

% We proceed by assuming that we have observed an arbitrary episode:
% \begin{align}
%   O'_j = [[l_k^{(b)}, r_k^{(b)}, l_k^{(e)}, r_k^{(e)}, i_j]^T, \vec{y}_k]^T.
% \end{align}
% We derive $p_k$ by considering the generating process that could have produced this episode.

For tractability, assume that the negative test bounding the start of the episode, on day $l_k^{(b)}-1$, is a true negative.
This assumption is reasonable because a positive test follows at $r_k^{(b)}$; therefore, $l_k^{(b)}-1$ is likely to be early in the infection episode when the test sensitivity is typically high.
True negatives occur with probability 1, and hence this test does not contribute to the likelihood.

As we assume that there are no false positives, the infection episode must span at least the period $[r^{(b)}_k, l^{(e)}_k]$, a period starting and ending with a positive test.
This includes all $t \in \sched'_k$ except $t = r_k^{(e)}+1$.
Therefore, the test results $\vec{y}_k$, except the test at $r_k^{(e)}+1$, are either true positives or false negatives.
This gives $t_+ = \sum_{l=1}^{m_k-1} y_{k,l}(t)$ true positives and $f_- = \sum_{l=1}^{m_k-1} (1 - y_{k,l}(t))$ false negatives.

Consider the negative test at $r_k^{(e)}+1$, the first negative after the start of the episode which may be either a true or false negative.
It is a false negative if and only if the episode ends at or after the test, \ie $E_j > r_k^{(e)}$.
By considering the case if whether this occurred or not and assuming $\psens$ is known and fixed, in \cref{sec:p-ia-dash} we show
\begin{align}
p_{ik}'
&\propto \sum_{b = l_k^{(b)}}^{r_k^{(b)}} S_{\vec{\theta}}(l_k^{(e)} - b + 1) - p_\text{sens} S_{\vec{\theta}}(r_k^{(e)} - b + 2).
\label{imperf-test:eq:pia-prime-constant}
\end{align}
Note that if $p_\text{sens} = 1$ then $p_{ik}' = p_{ik}$ (see \cref{perf-test:eq:pia}).

\subsection{Deriving $p'_{iu}$} \label{imperf-test:sec:modifying-p_iu}

We now modify $p_{iu}$ to form $p_{iu}' = \prob(O'_j = \emptyset \mid i_j = i, \vec{\theta})$ to take into account false negatives.
Several mechanisms for episodes being undetected were previously considered when deriving $p_{iu}$, we now consider the additional mechanisms arising due to false negatives.
Specifically, episode $j$ could be undetected if the first test after $b_j$ is a false negative and then there are no subsequent positive tests.

This false negative would occur at the first test after the infection episode begins, on day $b_j + \tau_{\sched_{i_j}}(b_j)$.
%% A false negative occurring requires that the episode has not yet ended but a negative still occurs.
The episode has not yet ended at the time of the test if $e_j = b_j + d_j - 1 \geq b_j + \tau_{\sched_{i_j}}(b_j)$, that is the duration of the infection $d_j \geq \tau_{\sched_{i_j}}(b_j) + 1$.
Conditional on the episode having not yet ended, the test result is negative with probability $1 - \psens$.

For there to be no subsequent positive tests, all tests up until day $e_j$ are false negatives.
We assume there is a negligible probability of missing an episode due to two false negative tests.
This is because that would require both a long episode, encompassing two test times, and for both these tests to be false negatives.
Therefore, an episode is undetected only if the episode ends before a second test.
Denote the number of days between $b_j$ and the test following the false negative as $\tau^2_{\sched_{i_j}}(b_j) \stackrel{\text{def}}{=} \tau_{\sched_{i_j}}(\tau_{\sched_{i_j}}(b_j) + 1)$.
The episode ends before this test if $d_j \leq \tau^2_{\sched_{i_j}}(b_j)$.

Therefore, this mechanism causes episode $j$ to be undetected if all the following conditions hold.
\begin{enumerate}
    \item The episode would have been detected considering only the mechanisms in \cref{sec:prob-undetected}. That is $\min(\sched_{i_j}) < b_j \leq T_{i_j}$ and $e_j \geq \tau_{\sched_{i_j}}(b_j) + b_j$.
    \item The episode ends in the interval $[\tau_{\sched_{i_j}}(b_j) + b_j, \tau^2_{\sched_{i_j}}(b_j) + b_j - 1]$.
      Note that the lower bound here is exactly the bound on $e_j$ in the previous condition.
      Equivalently, $\tau_{\sched_{i_j}}(b_j) + 1 \leq d_j \leq \tau^2_{\sched_{i_j}}(b_j)$.
    \item A false negative occurs on day $\tau_{\sched_{i_j}}(b_j) + b_j$. Conditional on the previous condition, this occurs with probability $1 - \psens$.
\end{enumerate}

In \cref{sec:p-iu-dash} we show that this gives:
\begin{align}
1 - p_{iu}'
=& \frac{1}{T} \sum_{b=\min(\sched_{i}) + 1}^{T_{i}} \left( p_\text{sens} S_{\vec{\theta}}(\tau_{\sched_{i}}(b) + 1) + (1 - p_\text{sens}) S_{\vec{\theta}}(\tau^2_{\sched_{i}}(b) + 1)\right).
\label{imperf-test:eq:pit-prime}
\end{align}

\section{The survival function} \label{sec:parameters-priors}

Next we specify the form and priors for $S_{\vec{\theta}}(t)$.
We parameterize $S$ in terms of the discrete-time hazard $S_{\vec{\theta}}(t) = \prod_{i=1}^{t-1} (1 - \lambda_{i})$; the lack of monotonicity or sum constraint on the hazard makes it an attractive parameterization for inference~\citep{heBayesian}.
Therefore, $\vec{\theta} = [\lambda_1, \dots, \lambda_{\dmax-1}]^T$, where $\dmax$ is the longest possible duration assumed as the maximum possible duration of the observed episodes, \ie $\dmax = \max_{k \in \set{D}} r^{(e)}_k - l^{(b)}_k + 1$.
We consider two priors (depicted in \cref{fig:priors}), with varying degrees of informativeness regarding $S_{\vec{\theta}(t)}$.
\begin{figure}
  \includegraphics{figures/output/prior_predictive_survival}
  \caption{%
    Prior predictive values of $S_{\vec{\theta}}$ for the two priors.
  }
  \label{fig:priors}
\end{figure}

The first prior for $S_{\vec{\theta}}$ is weakly informative, centred on prior estimates.
Specifically, we assume an independent prior distribution for each $\lambda_t$ of Beta(0.1, 1.9), which has mean 0.05 and little information.
The central 95\% probability mass of Beta(0.1, 1.9) is 0.00--0.47.
The central estimate, of 0.05, is in line with previous estimates that the median duration in the range 15--20 days~\citep{cevikShedding}.

The second is a strongly informative prior; it incorporates prior information from reliable estimates of $\lambda_t$ for $t < 20$.
We take a previous Bayesian analysis~\citep{blakeThesis} of data from The Assessment of Transmission and Contagiousness of COVID-19 in Contacts (ATACCC) study~\citep{hakkiOnset}, which tested individuals who had been exposed to infection daily up to a maximum of 20 days.
This ATACCC-based analysis produces posterior estimates of $\lambda_t$; however, it leads to a posterior distribution with positive correlation between $\lambda_t$ and $\lambda_{t'}$, especially for small $|t-t'|$.
Furthermore, the uncertainty in the prior estimates for $\lambda_t$ for $t\geq20$ are underestimated because they are based on extrapolation of the ATACCC data under strong model assumptions.
The following prior, based on the discrete Beta process prior~\citep{ibrahimBayesian,sunStatisticala}, incorporates both these aspects:
\begin{align}
  \logit \vec{h} &\dist \MNorm(\vec{\mu}_A, \matr{\Sigma}_A) \\
  \lambda_t &\dist \text{Beta}(\alpha_t, \beta_t) &t = 1, 2, \dots \\
  \alpha_t &= k_t h_t + \alpha_0 \\
  \beta_t &= k_t (1 - h_t) + \beta_0
\end{align}
where $k_t$, $\alpha_0$, and $\beta_0$ are hyperparameters; and $\vec{\mu}_A$ and $\matr{\Sigma}_A$ are posterior approximations of the ATACCC-based posterior (see \cref{sec:ataccc-prior} for details).
We use $\alpha_0 = 0.1$ and $\beta_0 = 1.9$ to match the weakly informative prior and the following form for $k_t$:
\begin{align}
k_t = \begin{cases}
  \expit(-0.4 * (t - 20)) &\text{for $t \leq 39$} \\
  0 &t > 39
\end{cases}.
\end{align}
This form reflects the subjective belief that $h_t$ is a good estimate of $\lambda_t$ for small $t$ but increasingly unreliable; specifically, it is large at 0, when ATACCC is reliable, but becomes small for $t \geq 20$.


\section{Simulations} \label{sec:simulation}


We use a simulation study to evaluate the performance of the method and the impact of the simplifying assumptions made.

\subsection{Setup}

We simulate a dataset of detected episodes that has the same characteristics as that in the CIS by the following procedure.
\begin{enumerate}
    \item Extract the test schedules for each individual who had at least one test during the period of interest.
    \item Draw an episode start time, $b_{j}$ for each individual uniformly at random between 2 July 2020 (100 days before the period where a detected episode would be included) and 6 December 2020 (the end of this period).
    \item Draw a duration of episode for their episode, $d_j$, based on a combination of previous estimates (described in \cref{sec:simulation-truth}). Then calculate the end of their infection episode, $e_{j} = b_{j} + d_i - 1$.
    \item Simulate the test results based on the test schedule, $b_{j}$, and $e_{j}$. A test on day $t$ between $b_{j}$ and $e_{j}$ (inclusive) is positive with probability $\psens$, where $\psens$ can vary with the time since infection, as defined below. All tests outside this interval are negative.
    \item Discard episodes where there are no positive tests (\ie undetected episodes) and then apply the inclusion criteria from \cref{sec:data}. Denote by $p$ the proportion of episodes that are retained.
    \item Of these remaining episodes, sample $\ndet = 4800$ to match the sample size of the true dataset. This is needed because in step 2 the entire cohort was infected, while in the real study only a (unknown) portion is infected.
    \item For this final set of episodes, calculate $(l_j^{(b)}, r_j^{b}, l_j^{(e)}, r_j^{(e)})$ by taking the day after the last negative prior to any positives, the first positive, the last positive, and the day before the negative following the last positive respectively.
\end{enumerate}

We simulate four scenarios for the test sensitivity.
The first three are constant, $\psens \in \{ 0.6, 0.8, 1.0 \}$.
The final scenario is a varying test sensitivity, which is more realistic~\citep{blakeThesis}.
Specifically we use the following form:
\begin{equation}
  v(t) = \begin{cases}
    0.9 - \frac{0.9-0.5}{50}t &t \leq 50 \\
    0.5 &t > 50
  \end{cases}
  \label{imperf-test:eq:variable-test-sensitivity}
\end{equation}
where $t$ is the number of days since the infection occurred.
We denote by $\psenss$ the true test sensitivity used in the simulation, with $\psenss = v$ indicating the varying test sensitivity.

For each scenario, we infer the survival function using the procedure proposed in this paper, assuming $\psens$ equals 0.6, 0.8, or 1.0.
That is, the value of $\psens$ used in inference is not necessarily $\psenss$, and we consider the impact of this misspecification.
We denote by $\psensi$ the assumed test sensitivity in inference.

If $\psenss = \psensi$, we refer to $\psens$ as being correctly specified; otherwise we refer to it as misspecified.
Note that if $\psenss = v$, then $\psens$ is always misspecified.
However, there is still some misspecification of the model to false negatives owing to the simplifying assumptions made.
The amount that the simplifying assumptions are violated increases as $\psens$ decreases.

Simulation was performed in R~4.2.0~\citep{R-4-2-0} using tidyverse~2.0.0~\citep{tidyverse}.
Inference was implemented in Stan via RStan~2.21.8~\citep{rstan2-21-8} using default settings.
Convergence was assessed using Rhat and ESS~\cite{vehtariRhat}, and all runs checked for divergent transitions.

We used a vague prior for $\ntot$, with $\mu = n_d / p$ and $r = 1$.

\subsection{Results}

When $\psens = 0.8$ and is correctly specified, the model recovers the true survival time well (see \cref{imperf-test:fig:constant-test-sensitivity}(B)).
The informative prior for $\vec\theta$, in comparison to the weakly informative prior, helps to overcome the misspecification due to the simplifying assumptions, moving the estimated survival function closer to its true survival time.
However, when $\psens = 0.6$, this is no longer the case (see \cref{imperf-test:fig:constant-test-sensitivity}(A)).
This is likely caused by too large a violation of the simplifying assumptions made in \cref{sec:false-negatives}.
\begin{figure}
  \includegraphics[width=\textwidth]{figures/output/sim-constant-sensitivity}
  \caption[Simulation study results with constant test sensitivity]{%
    Posterior (median and 95\% credible interval) of the survival time for the simulation study with a correctly specified test sensitivity.
    True survival time shown in black.
  }
  \label{imperf-test:fig:constant-test-sensitivity}
\end{figure}

Next, we considered the consequence of $\psens$ being misspecified and using the model combination prior, the better performing prior in the correctly specified case.
If the test sensitivity is misspecified then the survival time is biased.
If $\psensi < \psenss$, then the posterior estimate initially follows the true value but then separates (see \cref{imperf-test:fig:misspecified-test-sensitivity}(A)).
The number of episodes inferred to have truly ended by the first negative is too low, and hence the survival time is overestimated.
This effect dominates over the opposing bias of overestimating the number of undetected episodes.
The opposite occurs if $\psensi > \psenss$, although the posterior moves away from the truth earlier (see \cref{imperf-test:fig:misspecified-test-sensitivity}(C)).
\begin{figure}
  \includegraphics[width=\textwidth]{figures/output/sim-misspecified-sensitivity}
  \caption[Simulation study results with misspecified test sensitivity]{%
    Posterior (median and 95\% CrI) of the survival time.
    The true survival time is shown in black.
    All results use $\psenss = 0.8$, but $\psensi$ varies, as per panel labels.
  }
  \label{imperf-test:fig:misspecified-test-sensitivity}
\end{figure}

The results when $\psenss = v$ are similar to $\psenss = 0.8$ (see \cref{imperf-test:fig:variable-test-sensitivity}).
This suggests that the simplified model, with constant test sensitivity, is sufficient for recovering the true survival time.
% Estimating the test sensitivity is not possible without a more complex model, as discussed in \cref{imperf-test:sec:discussion}.
Therefore, we conclude that including a varying test sensitivity is not required for adequate inference, and apply it to the real CIS data in the next section.
\begin{figure}
  \includegraphics[width=\textwidth]{figures/output/sim-variable-sensitivity}
  \caption[Simulation study results with varying test sensitivity]{%
    Posterior (median and 95\% CrI) of the survival time.
    $\psensi$ is indicated on the panel label, $\psenss = 0.8$.
  }
  \label{imperf-test:fig:variable-test-sensitivity}
\end{figure}

\section{Application to the CIS data} \label{sec:CIS}

In this section we apply the approach described in this chapter to the CIS infection episode dataset.
Unlike in the simulation studies, an uninformative prior on $\ntot$ led to implausible estimates of the duration distribution.
The uninformative prior led to high posterior estimates of $\ntot$, and hence an implausibly large number of episodes with durations of less than five days.
Therefore, we used an informative prior for $\ntot$, $\ntot \sim \NBc(\mu\inform, r\inform)$ from pre-existing estimates of the total number of infections to give $\mu\inform$ and $r\inform$.
\citet{birrellRTM2} estimated the total number of infections in England over the time period we consider, with posterior mean \numprint{4136368} and standard deviation \numprint{27932}.
\todo{cite latest RTM paper}
% This model gives a posterior mean of \numprint{4136368} cumulative infections in England in the time period we consider, with a posterior standard deviation of \numprint{27932}~\citePersonalComms{Paul Birrell}.
Approximating this distribution as a negative binomial and scaling the mean to the size of the CIS cohort gives the prior $\mu\inform = 25132$ and $r\inform = 22047$.

With this prior, the model produces plausible estimates of the duration distribution (see \cref{imperf-test:fig:cis-estimates}).
The estimate (blue) has more long episodes than the ATACCC-based estimate used for the prior in \cref{sec:ataccc-prior} (red).

The qualitative increase in long episodes is robust to the choice of prior for $\ntot$, the assumed value for $\psens$ (see \cref{imperf-test:fig:cis-sensitivity}), and the choice of prior for the hazards, $\lambda_t$.
However, the survival proportion over the first 4 weeks is sensitive to these choices.
The estimate using a test sensitivity of 0.8 and $\NBc(\mu\inform, r\inform)$ give a median survival time most similar to the ATACCC-based estimate.
\begin{figure}
  \centering \includegraphics{figures/output/CIS_final}
  \caption{Duration estimates using CIS and ATACCC data}
  \label{imperf-test:fig:cis-estimates}
\end{figure}
\begin{figure}
  \thisfloatpagestyle{empty}
  \makebox[\textwidth][c]{\includegraphics[width=0.9\paperwidth]{figures/output/CIS_vary}}
  \caption{%
    Assessing prior sensitivity.
    (A-C) Changing $r$ when $\psens = 0.8$.
    (D-F) Changing $\psens$ when $r = r\inform$.
    A and D: median survival time, compared to ATACCC-based estimate (shown in red).
    B and E: $S_\theta(50)$.
    C and F: $S_\theta(t)$ for $t \in [1, 100]$.
  }
  \label{imperf-test:fig:cis-sensitivity}
\end{figure}


The estimates are sensitive to the choice of $r$, the strength of the prior on $\ntot$.
A low value for $r$, giving a very weak, almost uninformative, prior on $\ntot$ causes its posterior estimate to be much higher than the estimate from \citet{birrellRTM2}.
When increasing the prior's strength, the posterior estimate moves towards the prior smoothly, as expected (see \cref{imperf-test:fig:ntot}).
As discussed previously, the prior information is reliable for the first 2--3 weeks, notably including the median time.
The median using $r\inform$ matched the prior's median estimate and is a principled choice because it is based directly on the prior work \citet{birrellRTM2}.
Therefore, we recommend this estimate, which has a mean survival time of 21.2 days (95\% CrI: 20.5--21.9).
\begin{figure}
  \centering \includegraphics{figures/output/CIS_ntot}
  \caption[Sensitivity of $\ntot$'s posterior to its prior.]{How the posterior estimate of $\ntot$ changes with the value of $r$ in the prior on $\ntot$.}
  \label{imperf-test:fig:ntot}
\end{figure}


\section{Discussion} \label{sec:discussion}

This work is motivated by the challenge of exploiting data from the CIS, a unique long-running general population prevalence study conducted during the COVID-19 pandemic, to estimate the duration of SARS-CoV-2 infections.
To do so, we extended the survival analysis framework in \citet{heiseyModelling} to deal with the CIS design.
The result is new methodology to analyse doubly censored data with imperfect test sensitivity and undetected events, when these undetected events occur with an arbitrary pattern in known individuals.
We estimate a non-parametric discrete-time survival distribution in a fully Bayesian framework.

This CIS data is unique, and is likely to serve as a template for studies in future pandemics.
Our methodological framework is general and will likely be useful for these studies.
Furthermore, the simulation framework we developed can assist with the design of these studies.

We estimate 5.3\% (95\% CrI: 3.5--7.4\%) of episodes last 50 days or longer.
These episodes are unlikely to be infectious, but have implications for interpreting positive test results and for determining how long population positivity remains high following a large peak in infections.
Furthermore, we find a mean duration of 21.2 days (95\% CrI: 20.5--21.9), higher than previous studies; the incidence will often be overestimated if not incorporating this information.
The higher mean likely results from our larger estimates of the number of long episodes, which had not previously been well-known.

In this work we assume that infection episodes being at independent times.
The infectious nature of SARS-CoV-2 means that this is unlikely to be true, especially for individuals in the same household.
A simulation study (not shown), suggested that exponentially changing incidence does not substantially affect the survival estimates.
However, the interaction between individuals in the same household having both similar infection times and similar testing schedules may need to be considered, for example through incorporating random effects.

We also assume that the infection episodes are independent.
In particular, that any observation pattern can be observed for any infection.
This is not the case, as immunity means that an individual previously infected is less likely to be infected in the future.
Furthermore, we cannot detect two concurrent infections in the same individual.
The simulation study violated this assumption to a large extent: each individual could only have one infection.
Despite this, the method recovered the true survival function.
Therefore, it is unlikely that this assumption matters, possibly due to the large number of individuals without detected episodes.
Further work could explore alternative assumptions, \eg a ``full immunity'' assumption meaning each individual has at most one episode in the period.

We make several assumptions around how false negatives can occur (see \cref{sec:false-negatives}).
Our simulation study shows that these assumptions are reasonable, and do not substantially impact performance when $\psens$, the test sensitivity, is high.
However, when $\psens$ is low, these can lead to biased results.
A promising direction for future work would is relaxing these assumptions.
For example, inferring $\psens$ as a function of time since the episode began, similar to the generative model in \cref{sec:false-negatives}.

We incorporate two sources of external information into our estimates.
First, we use data from the ATACCC study as the basis for a prior for $\vec\theta$, the parameters of the survival function.
Second, we use information from \citet{birrellRTM2} to inform $\ntot$, the total number of infections.
Investigations into alternative study designs, which would allow similar studies to estimate the duration without these sources of information, would be valuable.
For example, assigning different testing schedules to each individual so that there are some short testing times or intensive follow-up of whole households once a single infection is detected in that household.
We provide a flexible simulation framework for enabling this work.

A final challenge this study faced is the use of the SRS.
As well as providing limited computational power, the SRS has practical limitations.
Internet access is unavailable when operating inside the SRS, and a lengthy approval process is required for software or data to be moved in or out of the environment.
Workflows and software amenable to operating in such an environment would speed future work similar to this study.

% What to be included
% \begin{itemize}
%   \item Most important assumptions and limitations, as future work: uniform and independent (OK in simulation); stuff around false negs (sim study here shows OK); issue with $n_k$; $\psens$ is too low
%     \item Sensitivity to priors: need for prior information in incidence, and where it could come from (RTM not available in real-time)

    
%     \item Results only applicable to this point in time
%     \item Design of CIS - random testing times, households
%     \item Value of complementary studies - can we run them in parallel? Eg followup a sub-sample of the positives.
%     \item Use of TREs, both computational and practical challenges (latter due to moving in / out)
    
 
% \end{itemize}

% $B_j$ is 2--3 days after the infection occurs~\citep{davisEstimating}, a small difference which we neglect in this study.

% We found that an informative prior on $\ntot$, the total number of infections that occurred, is required is for the bulk of the distribution to agree with previous work.
% In the simulation study, the uninformative prior on $\ntot$ performed well, even when the test sensitivity was misspecified.
% A possible explanation is that the true changes in the test sensitivity is significantly different to the simulated test sensitivity.
% In particular, the test sensitivity for long episodes may become much lower than 50\%.
% This allows many more episodes to be missed than the model assumes, and hence a higher $\ntot$ is required to explain the data.
% Furthermore, it violates assumptions made in \cref{sec:false-negatives}; this is similar to when the simulating using a too low test sensitivity.
% Violating these assumptions would lead to unpredictable inference results, perhaps those seen here.

% Alternative study designs could be useful in providing more information on the duration of infection episodes.
% For example, following up the first positive test for an infection episode with more frequent testing would allow the episode's end to be more accurately determined.

% We assumed a prior of constant incidence and that the time infection episodes begin is independent between individuals.
% Prevalence in the CIS data was approximately constant over the period of interest.
% A sensitivity analysis, where an epidemic in exponential growth or decline was simulated (not shown), showed minimal impact on survival estimates.
% An incorrect assumption of constant incidence can lead to biased estimates~\citep{degruttolaAnalysis}.
% Therefore, the assumption could be important in other contexts, especially if the incidence is changing rapidly.


% Ideally, the test sensitivity would be estimated simultaneously with the duration.
% However, this would require incorporating time-varying test sensitivity into the likelihood.
% If the current model, with a constant $\psens$, is used then the estimate of $\psens$ would be heavily informed be intermittent negatives (the first, constant term in \cref{imperf-test:eq:pia-prime}).
% These negatives will, in general, be further from the end of an episode than a randomly selected test.
% Therefore, the viral load will be higher and false negatives rarer.

% Estimating the test sensitivity excluding intermittent negatives is not possible because \cref{imperf-test:eq:pia-prime-constant,imperf-test:eq:pit-prime} are both monotonically decreasing in $\psens$; therefore, the likelihood always favours $\psens = 0$ (\ie no true positives).
% This aligns with the situation with singly interval censored, untruncated data, when stopping at the first observed time of the terminating event (which may be misclassified) means that the test sensitivity cannot be estimated~\citep[e.g.]{titmanMisclassify}.

% Estimating a time-varying $\psens$ with $S_\theta$ jointly may cause identifiability issues.
% Previous studies have avoided issues by including external information (such as a prior giving the magnitude) on $\psens$, or test results from later follow-up~\citep[and references therein]{piresIntervalMisclassify}.
% However, these studies use a constant $\psens$.
% Whether these methods are sufficient for the model to be identifiable with a time-varying $\psens$ needs further investigation.
% A simple parametric form of the test sensitivity, such as that proposed by \citet{brownBayesian}, may be sufficient to allow identifiability.
% In any case, the likelihood, especially $p_{iu}'$, would be substantially complicated by such an addition which may greatly increase the computational cost of inference.

% We provide a flexible simulation and inference framework for this type of analysis.
% This framework could be used to simulate alternate study designs.
% The design of CIS was created on a very short timescale in March 2020, in response to the pandemic's rapid spread.
% Therefore, it is likely that there are more efficient designs.
% These could be more cost-effective.
% Improved cost-effectiveness could allow more rapid response to potential pandemics because the threshold for policymakers to approve the study would be lower.
% A long-term, preparatory effort to develop a more efficient design in preparation for a future pandemic would be worthwhile.

% Other useful preparatory work would be developing reasonable priors.
% These could be based on seasonal viruses that are of the same family as those likely to cause future pandemics.
% For example, seasonal influenzas and coronaviruses.
% These would allow analyses such as those performed in these chapters to be performed more rapidly in a future pandemic scenario.

\bibliographystyle{agsm}

\bibliography{references}


\bigskip
\begin{center}
{\large\bf SUPPLEMENTARY MATERIAL}
\end{center}

\begin{description}

\item[Simulation package:] R-package to implement simulation studies similar to the one in this article is at \url{https://github.com/joshuablake/cisSimulation}.

\item[Analysis package:] R-package to generate Stan code to implement the inference method proposed in this article is at \url{https://github.com/joshuablake/cisDurationModel}.

\item[Code:] Scripts to reproduce the analyses in this article are available at \url{https://github.com/joshuablake/CIS_survival_analysis_sims}.

\item[Appendix:] Contains derivations and further details, as described in the main text (pdf).

\end{description}

\appendix

\section{Derivations of quantities in \cref{sec:inference}} \label{sec:derivations}

\subsection{Expressions in \cref{sec:inference}}

First, the derivation of $p(\vec{\theta} \mid \na)$:
\begin{align}
p(\vec{\theta} \mid \na)
&\propto p(\vec{\theta}) p(\na \mid \vec{\theta}) \\
&= p(\vec\theta) \sum_{\ntot= \ndet}^{\infty} p(\ntot, \na \mid \vec{\theta}) \\
&= p(\vec{\theta}) \sum_{\ntot=\ndet}^\infty p(\ntot \mid \vec{\theta}) p(\na \mid \ntot, \vec{\theta}) \\
&= p(\vec{\theta}) \sum_{\ntot=\ndet}^\infty p(\ntot \mid \vec{\theta}) \frac{\ntot!}{(\ntot - \ndet)!} \pnodet^{\ntot - \ndet} \prod_{k \in \set{D}} p_k &\text{by \cref{perf-test:eq:multinomial}} \\
&= p(\vec{\theta}) \left( \prod_{k \in \set{D}} p_k \right) \left( \sum_{\ntot=\ndet}^\infty p(\ntot \mid \vec{\theta}) \frac{\ntot!}{(\ntot - \ndet)!} \pnodet^{\ntot - \ndet} \right).
\intertext{For convenience, define the summation term as:}
\eta &= 
\sum_{\ntot=\ndet}^\infty p(\ntot \mid \vec{\theta}) \frac{\ntot!}{(\ntot - \ndet)!} \pnodet^{\ntot - \ndet}. \label{perf-test:eq:eta}
\end{align}

Next, we derive an analytical solution to $\eta$ (defined in \cref{perf-test:eq:eta}) assuming the prior $\ntot \dist \NBc(\mu, r)$, and that it is independent of $\vec{\theta}$, the parameters of the survival distribution.
Therefore, $p(\ntot \mid \vec{\theta}) = p(\ntot)$.
This assumption makes $\eta$ analytically tractable, allowing computationally feasible inference.

Putting a negative binomial prior on $\ntot$ is equivalent to the following gamma-Poisson composite; its use simplifies the derivation.
\begin{align}
\ntot \mid \lambda &\dist \Poi(\lambda) \\
\lambda &\dist \GamDist(a, b)
\end{align}
where $b = r / \mu$ and $a = r$.
Hence:
\begin{align}
\eta
&= \int \sum_{\ntot=\ndet}^\infty \frac{\ntot!}{(\ntot-\ndet)!} \pnodet^{\ntot-\ndet} p(\ntot \mid \lambda) p(\lambda) d\lambda &\text{$\lambda$ explicit}\\
&= \int \sum_{\ntot=\ndet}^\infty \frac{\ntot!}{(\ntot-\ndet)!} \pnodet^{\ntot-\ndet} \frac{\lambda^{\ntot} e^{-\lambda}}{\ntot!} p(\lambda) d\lambda &\ntot \dist \Poi\\
%&= \int \sum_{\ntot=\ndet}^\infty \frac{1}{(\ntot-\ndet)!} \pnodet^{\ntot-\ndet} \lambda^{\ntot-\ndet} \lambda^{\ndet} e^{-\lambda} p(\lambda) d\lambda \\
&= \int \lambda^{\ndet} e^{-\lambda} p(\lambda) \sum_{\nnodet=0}^\infty \frac{(\pnodet \lambda)^{\nnodet}}{\nnodet!} d\lambda &\nnodet = \ntot-\ndet\\
&= \int \lambda^{\ndet} e^{-\lambda} p(\lambda) e^{\lambda \pnodet} d\lambda &\text{Maclaurin series of $e$} \\
&= \int \lambda^{\ndet} e^{-\lambda(1 - \pnodet)} \frac{b^a}{\Gamma(a)} \lambda^{a-1} e^{-b\lambda} d\lambda &\lambda \dist \GamDist\\
&= \int \frac{b^a}{\Gamma(a)} \lambda^{a+\ndet-1} e^{-(b+1-\pnodet)\lambda} d\lambda \\
&= \frac{b^a}{\Gamma(a)} \frac{\Gamma(a+\ndet)}{(b+1-\pnodet)^{a+\ndet}} &\text{Gamma pdf}\\
&\propto (b+1-\pnodet)^{-(a+\ndet)} &\text{only $p_u$ depends on $\theta$}\\
&= (r/\mu + 1 - \pnodet)^{-(r+\ndet)} &\text{sub in $\mu$ and $r$}\\
&\propto(r + \mu (1- \pnodet))^{-(r+\ndet)}.
\end{align}

\subsection{Expressions in \cref{sec:prob-undetected}}

If $i_j = i_k$ then the event $O_j = \vec{\nu}_k$ occurs if and only if the episode starts in the interval $[l^{(b)}_k, r^{(b)}_k]$ and ends in the interval $[l^{(e)}_k, r^{(e)}_k]$.
 on $\vec{\theta}$ and $i_j = i_k$, this gives:
\begin{align}
p_{ik}
=& \prob \left( l_k^{(b)} \leq B_{j} \leq r_k^{(b)}, l_k^{(e)} \leq E_{j} \leq r_k^{(e)} \right) \\
=& \prob \left( l_k^{(e)} \leq E_{j} \leq r_k^{(e)} \mid l_k^{(b)} \leq B_{j} \leq r_k^{(b)} \right) \times\prob \left( l_k^{(b)} \leq B_{j} \leq r_k^{(b)} \right) \\
=& \sum_{b = l_k^{(b)}}^{r_k^{(b)}} \prob \left( l_k^{(e)} \leq E_{j} \leq r_k^{(e)} \mid B_{j} = b \right) \prob \left(B_{j} = b \right) \\
=& \sum_{b = l_k^{(b)}}^{r_k^{(b)}} \prob \left( l_k^{(e)} - b + 1 \leq D_{j} \leq r_k^{(e)} - b + 1 \right) \prob \left(B_{j} = b \right) &\text{by def of $D_{j}$} \\
=& \sum_{b = l_k^{(b)}}^{r_k^{(b)}} \left( S_{\vec{\theta}}(l_k^{(e)} - b + 1) - S_{\vec{\theta}}(r_k^{(e)} - b + 2) \right) \prob \left(B_{j} = b \right) &\text{by def of $S_{\vec{\theta}}$} \\
\propto& \sum_{b = l_k^{(b)}}^{r_k^{(b)}} \left( S_{\vec{\theta}}(l_k^{(e)} - b + 1) - S_{\vec{\theta}}(r_k^{(e)} - b + 2) \right)
\label{perf-test:eq:pia}
\end{align}
under the assumption of uniform probability of infection time.
This is the standard form of the likelihood for doubly interval censored data without truncation~\citep[e.g.][]{sunEmpirical}.

\subsection{Expression for $p_{u}$}

\begin{align}
  1 - p_u
  &= 1 - \sum_{i=1}^{\Ncis} \prob(O_j = \emptyset, i_j = i \mid \vec{\theta}) \\
  &= 1 - \sum_{i=1}^{\Ncis} \prob(O_j = \emptyset \mid i_j = i, \vec{\theta}) P(i_j = i \mid \vec{\theta}) \\
  &= 1 - \frac{1}{\Ncis}\sum_{i=1}^{\Ncis} \prob(O_j = \emptyset \mid i_j = i, \vec{\theta}) \\
  &= \frac{1}{\Ncis} \sum_{i=1}^{\Ncis} (1 - \prob(O_j = \emptyset \mid i_j = i, \vec{\theta}))
\end{align}

\section{Derivation of quantities in \cref{sec:false-negatives}}

\subsection{Expressions in \cref{imperf-test:sec:modifying-p_ia}} \label{sec:p-ia-dash}

We proceed by first considering whether $E_j > r_k^{(e)}$ is the case and conditioning on $B_j = b$.
Then, we combine the cases and remove the conditioning.

First, the case when $E_j \leq r_k^{(e)}$.
In this case, the test at $r_k^{(e)}+1$ is a true negative and the end of the episode is interval censored as in the previous chapter.
% In this case, the test at $r_k^{(e)} + 1$ is a true negative, as are all other tests not in $\sched'_{k}$.
The true negative occurs with probability 1, by the assumption of no false positives.
\begin{align}
&\prob(O'_j = \vec{\nu}_k', E_j \leq r_k^{(e)} \mid B_j = b, i_j = i_k, \psens, \vec{\theta}) \\
&= \prob(O'_j = \vec{\nu}_k', l_k^{(e)} \leq E_j \leq r_k^{(e)} \mid B_j = b, i_j = i_k, \psens, \vec{\theta}) &\text{the test at $l_k^{(e)}$ is positive} \\
&= \prob(O'_j = \vec{\nu}_k' \mid l_k^{(e)} \leq E_j \leq r_k^{(e)}, B_j = b, i_j = i_k, \psens, \vec{\theta}) \\
&\ \ \  \times \prob(l_k^{(e)} \leq E_j \leq r_k^{(e)} \mid B_j = b, i_j = i_k, \psens, \vec{\theta}) \\
&= p_\text{sens}^{t_+} (1 - p_\text{sens})^{f_-} \left( S_{\vec{\theta}}(l_k^{(e)} - b + 1) - S_{\vec{\theta}}(r_k^{(e)} - b + 2) \right)
\label{imperf-test:eq:ll-ei-lt-ri}
\end{align}

Second, the case when $E_j > r_k^{(e)}$.
In this case, the test at $r_k^{(e)}+1$ is a false negative, occurring with probability $(1 - p_\text{sens})$.
To avoid having to consider tests after $r_k^{(e)}$, which could greatly complicate the likelihood, we model this case as the episode being right censored at $r_k^{(e)}$.
Taking the same approach as before:
\begin{align}
&\prob(O'_j = \vec{\nu}_k', E_j > r_k^{(e)} \mid B_j = b, i_j = i_k, \psens, \vec{\theta}) \\
&= \prob(O'_j = \vec{\nu}_k' \mid E_j > r_k^{(e)}, B_j = b, i_j = i_k, \psens, \vec{\theta}) \\
  &\ \ \  \times \prob(E_j > r_k^{(e)} \mid B_j = b, i_j = i_k, \psens, \vec{\theta}) \\
&= p_\text{sens}^{t_+} (1 - p_\text{sens})^{f_-} (1 - p_\text{sens}) S_{\vec{\theta}}(r_k^{(e)} - b + 2)
\label{imperf-test:eq:ll-ei-gt-ri}
\end{align}

These expressions can now be used to derive $p'_{ik}$.
First, augment the data with $b$, and split into the cases just discussed, omitting the conditioning on $\psens$, $\vec{\theta}$, and $i_j = i_k$:
\begin{align}
p_{ik}'
=& \prob(O'_j = \vec{\nu}_k') \\
=& \sum_{b = l_k^{(b)}}^{r_k^{(b)}} \left( \prob(O'_j = \vec{\nu}_k', E_j \leq r_k^{(e)} \mid B_j = b) + \prob(O'_j = \vec{\nu}_k, E_j > r_k^{(e)} \mid B_j = b) \right) \prob(B_j = b). \\
\intertext{Now, substitute in \cref{imperf-test:eq:ll-ei-lt-ri,imperf-test:eq:ll-ei-gt-ri} and take out the common factor:}
=\ &  p_\text{sens}^{t_+} (1 - p_\text{sens})^{f_-} \\
 & \times \sum_{b = l_k^{(b)}}^{r_k^{(b)}} \left( S_{\vec{\theta}}(l_k^{(e)} - b + 1) - S_{\vec{\theta}}(r_k^{(e)} - b + 2) + (1 - p_\text{sens}) S_{\vec{\theta}}(r_k^{(e)} - b + 2) \right) \\ 
  & \times \prob(B_j = b \mid p_\text{sens}, \vec{\theta}) \\
=\ &  p_\text{sens}^{t_+} (1 - p_\text{sens})^{f_-} \\
  & \times \sum_{b = l_k^{(b)}}^{r_k^{(b)}} \left( S_{\vec{\theta}}(l_k^{(e)} - b + 1) - p_\text{sens} S_{\vec{\theta}}(r_k^{(e)} - b + 2) \right) \\
  & \times \prob(B_j = b \mid p_\text{sens}, \vec{\theta}).
\label{imperf-test:eq:pia-prime}
\end{align}
Note that if $p_\text{sens} = 1$ then $p_{ik}' = p_{ik}$ (see \cref{perf-test:eq:pia}).

If $\psens$ is fixed (\ie has a point prior) and $\prob(B_j = b \mid \psens, \vec{\theta}) \propto 1$ then:
\begin{align}
p_{ik}'
&\propto \sum_{b = l_k^{(b)}}^{r_k^{(b)}} S_{\vec{\theta}}(l_k^{(e)} - b + 1) - p_\text{sens} S_{\vec{\theta}}(r_k^{(e)} - b + 2).
\label{imperf-test:eq:pia-prime-constant}
\end{align}

\subsection{Expressions in \cref{imperf-test:sec:modifying-p_iu}} \label{sec:p-iu-dash}

The probability of one of the conditions that cause an episode to be missed (specified in \cref{imperf-test:sec:modifying-p_iu}) occurring, conditional on $B_j = b$ where $\min(\sched_{i_j}) < b \leq T_{i_j}$ is:
\begin{align}
&\prob \left(
    \tau_{\sched_{i_j}}(b) + 1 \leq D_j \leq \tau^2_{\sched_{i_j}}(b)
    \mid B_j = b, \vec{\theta} \right) (1 - \psens) \\
&= \left( S_{\vec{\theta}}(\tau_{\sched_{i_j}}(b) + 1) - S_{\vec{\theta}}(\tau^2_{\sched_{i_j}}(b) + 1) \right) (1 - \psens).
\end{align}
Summing over $b$, in the same way as \cref{perf-test:eq:piu}, gives:
\begin{align}
\zeta = (1 - p_\text{sens})\frac{1}{T} \sum_{b=\min(\sched_{i_j}) + 1}^{T_{i_j}} \left( S_{\vec{\theta}}(\tau_{\sched_{i_j}}(b) + 1) - S_{\vec{\theta}}(\tau^2_{\sched_{i_j}}(b) + 1) \right).
\label{imperf-test:eq:zeta}
\end{align}

$p_{iu}'$ is the probability of episode $i$ being undetected, considering both the previous and new mechanisms.
The previous and new mechanisms are mutually exclusive.
Hence, $p_{iu}'$ is the sum of these, $p_{iu}' = p_{iu} + \zeta$.
As previously, $1 - p_{iu}'$ is the required quantity.
\begin{align}
1 - p_{iu}'
=& 1 - p_{iu} - \zeta \\
% =& \frac{1}{T} \sum_{b = \min \sched_{i} + 1}^{T_{i_j}} S_{\vec{\theta}}(\tau_{\sched_{i}}(b) + 1) \\
% &- (1 - p_\text{sens})\frac{1}{T} \sum_{b=\min(\sched_{i_j}) + 1}^{T_{i_j}} \left( S_{\vec{\theta}}(\tau_{\sched_{i_j}}(b) + 1) - S_{\vec{\theta}}(\tau^2_{\sched_{i_j}}(b) + 1) \right) \\
=& \frac{1}{T} \sum_{b=\min(\sched_{i_j}) + 1}^{T_{i_j}} \left( p_\text{sens} S_{\vec{\theta}}(\tau_{\sched_{i_j}}(b) + 1) + (1 - p_\text{sens}) S_{\vec{\theta}}(\tau^2_{\sched_{i_j}}(b) + 1)\right).
\end{align}

\section{ATACCC-based prior} \label{sec:ataccc-prior}

Reliable estimates of $\lambda_t$ for $t$ up to around 20 are available from a prior analysis of data from The Assessment of Transmission and Contagiousness of COVID-19 in Contacts (ATACCC) study~\citep{hakkiOnset}, which tested individuals who had been exposed to infection daily up to a maximum of 20 days.
The infrequent testing in CIS means that there is a lack of information about short infection episodes, and hence we use these estimates as informative priors.

When constructing the prior, two aspects need consideration.
Firstly, the model structure from the ATACCC-based analysis leads to its posterior distribution having a positive correlation between $\lambda_t$ and $\lambda_{t'}$, especially for small $|t-t'|$.
The prior used in this analysis should preserve this correlation.
Secondly, the uncertainty in the prior estimates for $\lambda_t, t\geq20$ are underestimated because they are based on extrapolation of the ATACCC data using strong model assumptions.

We first approximate the previous posterior estimate of the hazard as $\logit{\vec{h}} \dist \MNorm(\vec{\mu}_A, \matr{\Sigma}_A)$ where $\vec{h}$ is the hazard, and $\vec{\mu}_A$ and $\matr{\Sigma}_A$ are the mean and covariance matrix estimated using samples of $\logit{\vec{h}}$ from the previous study's posterior.
Using a multivariate normal, as opposed to multiple univariate distribution for each $h_t$, preserves the correlation between the hazards.
% The logit transformation maps from the $[0, 1]$ interval to the full real line.
The approximation is very good (not shown).

Having approximated the estimate as a multivariate normal, we add additional uncertainty using a discrete Beta process.
The discrete Beta process prior~\citep{ibrahimBayesian,sunStatisticala} generalises the form of prior used in the weakly informative case by allowing the central estimate of the hazard to vary over time.
It is:
\begin{align}
  \logit \vec{h} &\dist \MNorm(\vec{\mu}_A, \matr{\Sigma}_A) \\
  \lambda_t &\dist \text{Beta}(\alpha_t, \beta_t) &t = 1, 2, \dots \\
  \alpha_t &= k_t h_t + \alpha_0 \\
  \beta_t &= k_t (1 - h_t) + \beta_0
\end{align}
where $k_t$, $\alpha_0$, and $\beta_0$ are hyperparameters.
An intuition for what this distribution represents derives from a conjugate model for $\lambda_t$ with a beta prior and a binomial likelihood.
If $\lambda_t$ is given the prior distribution $\text{Beta}(\alpha_0, \beta_0)$, and we then have $k_t$ observations with $k_t h_t$ successes, then the posterior distribution for $\lambda_t$ is $\text{Beta}(\alpha_t, \beta_t)$ (as defined above).

$k_t$ reflects the subjective belief that $h_t$ is a good estimate of $\lambda_t$ for small $t$ but increasingly unreliable.
\begin{align}
k_t = \begin{cases}
  \expit(-0.4 * (t - 20)) &\text{for $t \leq 39$} \\
  0 &t > 39
\end{cases}.
\end{align}

\section{Distribution of duration used in simulation} \label{sec:simulation-truth}

The simulation requires a distribution of the duration of detectability.
We modify the ATACCC-based duration estimate from \citet[chapter 4]{blakeThesis} with an inflated tail to be consistent with the CIS.
The tail inflation uses a simple survival analysis and the CIS data.

This analysis assumes the initiating event is known, and equal to the episode’s detection time, $j_j^{(b)}$.
It assumes the final event is interval censored between the time of the final positive test and the subsequent negative test, or right censored if a negative test has not yet been observed.
A flexible, spline-based form is used for the baseline survival function~\citep{roystonSTPM,roystonFlexible} with covariates introduced via proportional odds.
By not accounting for either the undetected infections or the interval censoring of the initiating event, this analysis has competing biases which makes them hard to interpret~\citep{cisMethodsONS}.

To form the duration distribution used in the simulation, we combine the two estimates.
The pdf over the first 30 days is proportional to the ATACCC estimate, with the rest proportional to this CIS-based estimate.
Denote by $f_A(t)$ the ATACCC-based distribution function and $f_C(t)$ that from the CIS-based estimates just derived.
Then define:
$$
f_S'(t) = \begin{cases}
	f_A(t) &t \leq 30 \\
	f_C(t) &t > 30
\end{cases}
$$
Then the distribution used in the simulation is the normalised version of this: $f_S(t) = f'_S(t)/\sum_i f_S'(i)$.
% These curves and the combined curve are compared in \cref{perf-test:fig:duration-dist}.
Episode $j$'s duration of detectability is then a draw from this distribution.

\end{document}
