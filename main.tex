\documentclass[12pt]{article}
\usepackage{amsmath}
\usepackage{graphicx}
\usepackage{enumerate}
% \usepackage{natbib}
\usepackage{url} % not crucial - just used below for the URL 

%\pdfminorversion=4
% NOTE: To produce blinded version, replace "0" with "1" below.
\newcommand{\blind}{0}

% DON'T change margins - should be 1 inch all around.
\addtolength{\oddsidemargin}{-.5in}%
\addtolength{\evensidemargin}{-1in}%
\addtolength{\textwidth}{1in}%
\addtolength{\textheight}{1.7in}%
\addtolength{\topmargin}{-1in}%

%% ABOVE THIS LINE IS JASA TEMPLATE
%% BELOW THIS LINE IS MY STUFF

\usepackage[utf8]{inputenc}
\usepackage{amssymb}
\usepackage{floatpag}
\usepackage{bm}
% \usepackage{algorithm2e}
\usepackage[unicode,psdextra]{hyperref}
\usepackage[nameinlink]{cleveref}
\usepackage{csquotes}
\usepackage[T1]{fontenc}
\usepackage{textcomp} % provide symbols
\usepackage{xr}
\usepackage{afterpage}
\usepackage{caption}
\usepackage{numprint}
\npfourdigitnosep
\npdecimalsign{.}
% \usepackage{microtype}
\usepackage{makecell}

% Generic maths commands
\def\reals{\mathbb{R}}
\def\nats{\mathbb{N}}
\def\sampSpace{\mathcal{X}}
\def\dist{\sim}
\DeclareMathOperator{\E}{\mathbb{E}}
\DeclareMathOperator{\V}{\mathbb{V}}
\DeclareMathOperator{\I}{\mathbb{I}}
\DeclareMathOperator{\prob}{\mathbb{P}}
\DeclareMathOperator{\p}{\pi}
\DeclareMathOperator{\var}{\mathbb{V}}
\DeclareMathOperator{\indicator}{\mathbb{I}}
\DeclareMathOperator{\cov}{Cov}
\DeclareMathOperator{\cor}{Cor}
\DeclareMathOperator{\logit}{logit}
\DeclareMathOperator{\Ber}{Bernoulli}
\DeclareMathOperator{\Bin}{Binomial}
\DeclareMathOperator{\Poi}{Poisson}
\DeclareMathOperator{\BetaDist}{Beta}
\DeclareMathOperator{\Exponential}{Exponential}
\DeclareMathOperator{\NBr}{NegBin}
\newcommand{\NBc}{\NBr_{c}}
\newcommand{\NBs}{\NBr_{s}}
\DeclareMathOperator{\BB}{BetaBin}
\DeclareMathOperator{\GamDist}{Gamma}
\DeclareMathOperator{\MN}{Multinomial}
\DeclareMathOperator{\N}{N}
\DeclareMathOperator{\MNorm}{N}
\DeclareMathOperator{\LN}{LN}
\DeclareMathOperator{\LKJ}{LKJ}
\DeclareMathOperator{\expit}{expit}
\newcommand\matr{\bm}
\newcommand\set{\mathcal}
\renewcommand{\vec}[1]{\bm{#1}}
\newcommand{\ssep}{:}
\DeclareMathOperator*{\argmax}{arg\,max}

% Thesis-specific maths commands
\newcommand{\dmax}{d_\text{max}}
\newcommand{\psens}{p_\text{sens}}
\newcommand{\psenss}{p_\text{sens}^{(s)}}
\newcommand{\psensi}{p_\text{sens}^{(i)}}
\newcommand{\ntot}{n_\text{tot}}
\newcommand{\ndet}{n_\text{d}}
\newcommand{\nnodet}{n_\text{u}}
\newcommand{\pnodet}{p_\text{u}}
\newcommand{\Npop}{N_\text{pop}}
\newcommand{\Ncis}{N_\text{CIS}}
\newcommand{\ncis}{\vec{n_\text{CIS}}}
\newcommand{\na}{\vec{n}_\text{obs}}
\newcommand{\pcis}{\vec{p_\text{CIS}}}
\newcommand{\sched}{\mathbb{T}}
\newcommand{\nsched}{n_{\sched}}
\newcommand{\inform}{{_{\text{inform}}}}


%% Bibliography
\usepackage[authordate-trad,backend=biber]{biblatex-chicago}
\addbibresource{references.bib}


\begin{document}

%\bibliographystyle{natbib}

\def\spacingset#1{\renewcommand{\baselinestretch}%
{#1}\small\normalsize} \spacingset{1}


%%%%%%%%%%%%%%%%%%%%%%%%%%%%%%%%%%%%%%%%%%%%%%%%%%%%%%%%%%%%%%%%%%%%%%%%%%%%%%

\if1\blind
{
  \title{\bf Title}
  \author{Author 1\thanks{
    The authors gratefully acknowledge \textit{please remember to list all relevant funding sources in the unblinded version}}\hspace{.2cm}\\
    Department of YYY, University of XXX\\
    and \\
    Author 2 \\
    Department of ZZZ, University of WWW}
  \maketitle
} \fi

\if0\blind
{
  \bigskip
  \bigskip
  \bigskip
  \begin{center}
    {\LARGE\bf Title}
\end{center}
  \medskip
} \fi

\bigskip
\begin{abstract}
The text of your abstract. 200 or fewer words.
\end{abstract}

\noindent%
{\it Keywords:}  3 to 6 keywords, that do not appear in the title
\vfill

\newpage
\spacingset{1.9} % DON'T change the spacing!
\section{Introduction}
\label{sec:intro}

Body of paper.  Margins in this document are roughly 0.75 inches all
around, letter size paper.

\begin{figure}
\begin{center}
\includegraphics[width=3in]{fig1.pdf}
\end{center}
\caption{Consistency comparison in fitting surrogate model in the tidal
power example. \label{fig:first}}
\end{figure}

\begin{table}
\caption{D-optimality values for design $X$ under five different scenarios.  \label{tab:tabone}}
\begin{center}
\begin{tabular}{rrrrr}
one & two & three & four & five\\\hline
1.23 & 3.45 & 5.00 & 1.21 & 3.41 \\
1.23 & 3.45 & 5.00 & 1.21 & 3.42 \\
1.23 & 3.45 & 5.00 & 1.21 & 3.43 \\
\end{tabular}
\end{center}
\end{table}

\begin{itemize}
\item Note that figures and tables (such as Figure~\ref{fig:first} and
Table~\ref{tab:tabone}) should appear in the paper, not at the end or
in separate files.
\item In the latex source, near the top of the file the command
\verb+\newcommand{\blind}{1}+ can be used to hide the authors and
acknowledgements, producing the required blinded version.
\item Remember that in the blind version, you should not identify authors
indirectly in the text.  That is, don't say ``In Smith et. al.  (2009) we
showed that ...''.  Instead, say ``Smith et. al. (2009) showed that ...''.
\item These points are only intended to remind you of some requirements.
Please refer to the instructions for authors
at \url{http://amstat.tandfonline.com/action/authorSubmission?journalCode=uasa20&page=instructions#.VFkk7fnF_0c}
\item For more about ASA\ style, please see \url{http://journals.taylorandfrancis.com/amstat/asa-style-guide/}
\item If you have supplementary material (e.g., software, data, technical
proofs), identify them in the section below.  In early stages of the
submission process, you may be unsure what to include as supplementary
material.  Don't worry---this is something that can be worked out at later stages.
\end{itemize}

\section{Methods}
\label{sec:meth}
Don't take any of these section titles seriously.  They're just for
illustration.



\section{Verifications}
\label{sec:verify}

\autocite{abbottCISincidence}

This section will be just long enough to illustrate what a full page of
text looks like, for margins and spacing.

\addtolength{\textheight}{.5in}%


\cite{Campbell02}, \cite{Schubert13}, \cite{Chi81}


The quick brown fox jumped over the lazy dog.
The quick brown fox jumped over the lazy dog.
The quick brown fox jumped over the lazy dog.
The quick brown fox jumped over the lazy dog.
{\bf With this spacing we have 26 lines per page.}
The quick brown fox jumped over the lazy dog.
The quick brown fox jumped over the lazy dog.
The quick brown fox jumped over the lazy dog.
The quick brown fox jumped over the lazy dog.
The quick brown fox jumped over the lazy dog.

The quick brown fox jumped over the lazy dog.
The quick brown fox jumped over the lazy dog.
The quick brown fox jumped over the lazy dog.
The quick brown fox jumped over the lazy dog.
The quick brown fox jumped over the lazy dog.
The quick brown fox jumped over the lazy dog.
The quick brown fox jumped over the lazy dog.
The quick brown fox jumped over the lazy dog.
The quick brown fox jumped over the lazy dog.
The quick brown fox jumped over the lazy dog.

The quick brown fox jumped over the lazy dog.
The quick brown fox jumped over the lazy dog.
The quick brown fox jumped over the lazy dog.
The quick brown fox jumped over the lazy dog.
The quick brown fox jumped over the lazy dog.
The quick brown fox jumped over the lazy dog.
The quick brown fox jumped over the lazy dog.
The quick brown fox jumped over the lazy dog.
The quick brown fox jumped over the lazy dog.
The quick brown fox jumped over the lazy dog.

%new margin nere

The quick brown fox jumped over the lazy dog.
The quick brown fox jumped over the lazy dog.
The quick brown fox jumped over the lazy dog.
The quick brown fox jumped over the lazy dog.
The quick brown fox jumped over the lazy dog.
The quick brown fox jumped over the lazy dog.
The quick brown fox jumped over the lazy dog.
The quick brown fox jumped over the lazy dog.
The quick brown fox jumped over the lazy dog.
The quick brown fox jumped over the lazy dog.

\addtolength{\textheight}{-.3in}%


The quick brown fox jumped over the lazy dog.
The quick brown fox jumped over the lazy dog.
The quick brown fox jumped over the lazy dog.
The quick brown fox jumped over the lazy dog.
The quick brown fox jumped over the lazy dog.
The quick brown fox jumped over the lazy dog.
The quick brown fox jumped over the lazy dog.
The quick brown fox jumped over the lazy dog.
The quick brown fox jumped over the lazy dog.
The quick brown fox jumped over the lazy dog.

The quick brown fox jumped over the lazy dog.
The quick brown fox jumped over the lazy dog.
The quick brown fox jumped over the lazy dog.
The quick brown fox jumped over the lazy dog.
The quick brown fox jumped over the lazy dog.
The quick brown fox jumped over the lazy dog.
The quick brown fox jumped over the lazy dog.
The quick brown fox jumped over the lazy dog.
The quick brown fox jumped over the lazy dog.
The quick brown fox jumped over the lazy dog.

The quick brown fox jumped over the lazy dog.
The quick brown fox jumped over the lazy dog.
The quick brown fox jumped over the lazy dog.
The quick brown fox jumped over the lazy dog.
The quick brown fox jumped over the lazy dog.
The quick brown fox jumped over the lazy dog.
The quick brown fox jumped over the lazy dog.
The quick brown fox jumped over the lazy dog.
The quick brown fox jumped over the lazy dog.
The quick brown fox jumped over the lazy dog.

The quick brown fox jumped over the lazy dog.
The quick brown fox jumped over the lazy dog.
The quick brown fox jumped over the lazy dog.
The quick brown fox jumped over the lazy dog.
The quick brown fox jumped over the lazy dog.
The quick brown fox jumped over the lazy dog.
The quick brown fox jumped over the lazy dog.
The quick brown fox jumped over the lazy dog.
The quick brown fox jumped over the lazy dog.
The quick brown fox jumped over the lazy dog.

The quick brown fox jumped over the lazy dog.
The quick brown fox jumped over the lazy dog.
The quick brown fox jumped over the lazy dog.
The quick brown fox jumped over the lazy dog.
The quick brown fox jumped over the lazy dog.
The quick brown fox jumped over the lazy dog.
The quick brown fox jumped over the lazy dog.
The quick brown fox jumped over the lazy dog.
The quick brown fox jumped over the lazy dog.
The quick brown fox jumped over the lazy dog.

The quick brown fox jumped over the lazy dog.
The quick brown fox jumped over the lazy dog.
The quick brown fox jumped over the lazy dog.
The quick brown fox jumped over the lazy dog.
The quick brown fox jumped over the lazy dog.
The quick brown fox jumped over the lazy dog.
The quick brown fox jumped over the lazy dog.
The quick brown fox jumped over the lazy dog.
The quick brown fox jumped over the lazy dog.
The quick brown fox jumped over the lazy dog.

The quick brown fox jumped over the lazy dog.
The quick brown fox jumped over the lazy dog.
The quick brown fox jumped over the lazy dog.
The quick brown fox jumped over the lazy dog.



\section{Conclusion}
\label{sec:conc}


\bigskip
\begin{center}
{\large\bf SUPPLEMENTARY MATERIAL}
\end{center}

\begin{description}

\item[Title:] Brief description. (file type)

\item[R-package for  MYNEW routine:] R-package MYNEW containing code to perform the diagnostic methods described in the article. The package also contains all datasets used as examples in the article. (GNU zipped tar file)

\item[HIV data set:] Data set used in the illustration of MYNEW method in Section~ 3.2. (.txt file)

\end{description}

\section{BibTeX}

We hope you've chosen to use BibTeX!\ If you have, please feel free to use the package natbib with any bibliography style you're comfortable with. The .bst file agsm has been included here for your convenience. 

\printbibliography

\end{document}
